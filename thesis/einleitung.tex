\chapter{Einleitung}
\label{cha:Einleitung}


\section{Voraussetzungen}
Vorausgegangen sind dieser Arbeit \ua die Zertifizierung als ISTQB Certified Tester Foundation Level und mehrere Projekte im SAP-Mobilbereich.

Bisher waren Apps mit Zugriff auf SAP-Systeme vollständige Eigenentwicklungen auf Basis externer Frameworks -- SAP bietet mit SAPUI5, Gateway und Mobile Platform neue Möglichkeiten zur Entwicklung von geräteübergreifenden Anwendungen aus einer Hand. Wir wollen sie nutzen, um bestehende Individualsoftware auf ABAP-Basis mobil nutzbar zu machen.

Gleichzeitig automatisieren wir den Entwicklungs- und
Auslieferungsprozess. Wir evaluieren entsprechende Open Source Tools und geben Hinweise auf  Best Practices. Die SAP-Grundlagen erweitern wir um aktuelle Vorgehensmodelle und Entwicklungsmethoden im Sinne der testgetriebenen Entwicklung.


\section{Problemstellung}
Viele Projekte der abat\,AG arbeiten agil \zB per Scrum. Zur Organisation ist ein umfangreiches Projektmanagement-Tool als ABAP-Eigenentwicklung vorhanden.

Dieses enthält allerdings weder ein Scrum-Board noch eine mobile Ansicht --
schneller Zugriff auf wichtige Funktionen ist unterwegs unmöglich. Die Aufgaben
können nur am PC mit Intranet-Zugang bearbeitet werden.

Der aktuelle Workflow: Ausdrucken der einzelnen Aufgaben, anpinnen, manuell
verschieben und parallel per Scrum-Transaktion in das SAP-System übertragen.
Dies gilt es mit aktuellen Technologien zu vereinfachen.


\section{Zielsetzung}
Im Rahmen der Bachelorarbeit wird eine geräteübergreifende App entwickelt, die
das Scrum-Board visualisiert und den Zugriff auf Projektdaten schneller und
einfacher gestaltet.

Während der testgetriebenen Entwicklung sollen aktuelle Technologien und Tools zum Einsatz
kommen.

In Zukunft sollen die Projektaufgaben mit Zusatzinformationen auf einem Smartphone oder
Tablet angezeigt und bearbeitet werden. Ein typischer Vorgang in dieser App wird
beispielsweise die Statusänderung von Aufgaben -- Diese kann per Drag and
Drop deutlich schneller erledigt werden.

Der bedeutendste Vorteil ergibt sich aus der ständigen Verfügbarkeit des
Projektstatus:
Das Scrum-Board muss nicht mehr physisch vorhanden sein, ein Blick in die App
genügt.
Der umständliche Zugriff über die alte, sehr umfangreiche SAP-Transaktion ist nur noch selten notwendig.


\section{Erkenntnisinteresse}
Besonders hervorzuheben ist die Kombination der verschiedenen Aspekte und
Vorgehen:

\begin{enumerate}
	\item Entwicklung einer aktuellen SAPUI5-App für ein bereits vorhandenes
	Altsystem auf ABAP-Basis.
	\item Die Integration des neuen NetWeaver Gateways und der entsprechenden
	OData-Services.
	\item Nutzung des Frameworks für Logon- und	Offline-Funktionen.
	\item Erstellung von Testfällen anhand der Spezifikation.
	\item Zuverlässigkeit vorhandener Features nach Updates durch automatische
	Regressionstests.
	\item Aufbau der Open-Source-CI-Toolchain für eine SAPUI5-Entwicklung.
	\item Kontinuierliche Bereitstellung neuer App-Versionen für
	verschiedene Gerätetypen.
\end{enumerate}
Für alle folgenden Projekte werden diese Aspekte essentiell sein: Es gilt eine
entsprechende Toolchain zu erproben und zu etablieren, um
Softwarequalität und Erfüllung der Spezifikation nachhaltig zu gewährleisten.