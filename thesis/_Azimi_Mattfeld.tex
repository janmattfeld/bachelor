\documentclass[bachelor,german]{hgbthesis}
% Zulässige Class Options: 
%   Typ der Arbeit: diplom, master (default), bachelor, praktikum 
%   Hauptsprache: german (default), english
%%------------------------------------------------------------

\usepackage{microtype} % Verhindert under-/overfull hboxes
\usepackage[nolist,nohyperlinks]{acronym} % Abkürzungen
\usepackage{booktabs} % Schönere Tabellen \toprule \midrule \bottomrule
\usepackage{wrapfig} % Grafiken im Rand
\usepackage{picins} % Grafik neben Text
\captionsetup[parpic]{
	skip=20pt
} % Setzt Abbildungs Caption in die Mitte des Whitespace

\usepackage[outputdir=.texpadtmp]{minted}
\usepackage{minted} % Schönere Code Listings


% !TeX spellcheck = de_DE
% !TeX encoding = utf8
% !TeX root = _Azimi_Mattfeld.tex
% !TeX TXS-program:latex = txs://pdflatex -synctex=1 -shell-escape -interaction=nonstopmode %.tex
% !BIB program = biber % biber statt bibtex automatisch auswählen

\graphicspath{{images/}}    	% name of directory containing the images
\logofile{logo_hs_bremerhaven}	% name of PDF, remove or use \logofile{} for no logo
\bibliography{literatur}  		% name of the BibTeX (.bib) file

% Mehrere Autoren
\newcommand\autsection[2]{%
	\section[#1~/ {\normalfont\small\itshape#2}]{#1}}

\newcommand\autsubsection[2]{%
	\subsection[#1~/ {\normalfont\small\itshape#2}]{#1}}


\newcommand{\myTitle}{Testgetriebene Entwicklung und kontinuierliche Integration mit der SAP Mobile Platform\xspace} 
\newcommand{\myKeywords}{SAPUI5, OData,	Continuous Integration, Test-Driven-Development\xspace}
\newcommand{\mySubtitle}{\xspace}
\newcommand{\myDegree}{Bachelor of Science\xspace}
\newcommand{\myName}{Maximilian Azimi, Jan-Henrich Mattfeld\xspace}
\newcommand{\myMatrikel}{29287, 29866\xspace}
\newcommand{\myProf}{Prof. Dr. Karin Vosseberg}
\newcommand{\myCourse}{\xspace}
\newcommand{\myOtherProf}{\xspace}
\newcommand{\mySupervisor}{mgr\,inż. Alfred Schmidt\xspace}
\newcommand{\myFaculty}{\xspace}
\newcommand{\myDepartment}{Fachbereich\,2\xspace}
\newcommand{\myUni}{Hochschule Bremerhaven\xspace}
\newcommand{\myUniAddress}{An der Karlstadt 8\xspace\\27586 Bremerhaven\xspace}
\newcommand{\myLocation}{Bremerhaven\xspace}
\newcommand{\myTime}{März 2014\xspace}
\newcommand{\myVersion}{Version 1.0\xspace}

\begin{acronym}[AJAX]
	\acro{TDD}{Test Driven Development}
	\acro{CI}{Continuous Integration}
	\acro{DOM}{Document Object Model}
	\acro{AJAX}{Asynchronous JavaScript and XML}
	\acro{RFC}{Remote Function Call}
	\acro{BAPI}{Business Application Programming Interface}
	\acro{SDK}{Software Development Kit}
	\acro{API}{Application Programming Interface}
	\acro{OData}{Open Data Protocol}
	\acro{REST}{Representational State Transfer}
	\acro{ABAP}{Advanced Business Application Programming}
	\acro{BOR}{Business Object Repository}
	\acro{ATDD}{Acceptance Test Driven Development}
	\acro{BDD}{Behavior Driven Development}
	\acro{BSP}{Business Server Pages}
	\acro{IDE}{Integrierte Entwicklungsumgebung}
	\acro{MDM}{Mobile Device Management}
	\acro{MAM}{Mobile Application Management}
	\acro{EMM}{Enterprise Mobility Management}
	\acro{OSP}{Microsoft Open Specification Promise}
	\acro{EDM}{Entity Data Model}
	\acro{MVC}{Model View Controller}
	\acro{eCATT}{extended Computer Aided Test Tool}
\end{acronym}

%%%----------------------------------------------------------
\begin{document}
%%%----------------------------------------------------------

% Einträge für ALLE Arbeiten: --------------------------------
\title{\myTitle}
\author{\myName}
\studiengang{Wirtschaftsinformatik}
\studienort{Bremerhaven}
\abgabedatum{2015}{03}{23}	% {YYYY}{MM}{DD}

%%%----------------------------------------------------------
\frontmatter
\maketitle
\tableofcontents


\listoffigures
\addcontentsline{toc}{chapter}{Abbildungsverzeichnis}

%\addcontentsline{toc}{chapter}{Tabellenverzeichnis}
%\listoftables
%
%\addcontentsline{toc}{chapter}{Listing}
%\listoflistings
%%%----------------------------------------------------------
%
\chapter{Autorenverzeichnis}

\begin{tabular}{lr}
	\toprule Kapitel & Autor\\
	\midrule 
	1. Einleitung & Beide?\\
	\midrule 
	2.1 Was ist testgetriebene Entwicklung? & Jan-Henrich Mattfeld\\
	2.2 SAP Mobile & Jan-Henrich Mattfeld\\
	2.3 Neues Backend: SAP Gateway & Max\\
	\midrule 
	3.1 App-Spezifikation & Jan-Henrich Mattfeld \\	 
	3.2.1 Steuerung und Planung & Jan-Henrich Mattfeld\\
	3.2.2 Durchführung & Beide\\
	\midrule 
	4.1 Backend & Max \\
	4.2.1 Paketmanager & Jan-Henrich Mattfeld\\
	4.2.2 ESLint & Jan-Henrich Mattfeld\\
	4.2.3 Karma & Jan-Henrich Mattfeld\\
	4.2.4 Hybrid-AppperPhoneGapBuild & Max\\
	4.2.5 JSTaskRunnerGrunt & Max\\
	4.2.6 Jenkins & Max\\
	4.2.7 Zusammenfassung & Max\\
	4.3 UI5-App & Jan-Henrich Mattfeld \\
	\midrule 
	5. Schlussfolgerungen & Beide \\
	\bottomrule 
\end{tabular}

\chapter{Kurzfassung}

Mit aktuellen Tools und Frameworks wie SAPUI5, NW\,Gateway und SMP bietet die
SAP neue Möglichkeiten zur Entwicklung von geräteübergreifenden mobilen
Anwendungen. Diese wollen wir nutzen, um Individualsoftware der abat\,AG mobil
nutzbar zu machen. Gleichzeitig sollen der Entwicklungs- und
Auslieferungsprozess automatisiert und entsprechende Tools erprobt werden.
		
%\include{abstract}			

%%%----------------------------------------------------------
\mainmatter         % Hauptteil (ab hier arab. Seitenzahlen)
%%%----------------------------------------------------------

\chapter{Einleitung}
\label{cha:Einleitung}

\section{Einleitung}
Die Ergebnisse entstehen im Rahmen unserer Bachelorarbeit im Fach
Wirtschaftsinformatik an der Hochschule Bremerhaven, die voraussichtlich im
Februar 2015 abgeschlossen sein wird. Vorausgegangen sind dieser u.\,a.
die Zertifizierung als ISTQB Certified Tester Foundation Level und mehrere
Projekte im SAP-Mobilbereich.

\section{Problemstellung}
Viele Projekte der abat\,AG arbeiten agil z.\,B. per Scrum.
Hierzu ist ein umfangreiches Projektmanagement-Tool als ABAP-Eigenentwicklung
vorhanden.

Dieses enthält allerdings weder ein Scrum-Board noch eine mobile Ansicht --
schneller Zugriff auf wichtige Funktionen ist unterwegs unmöglich. Die Aufgaben
können nur am PC mit Intranet-Zugang bearbeitet werden.

Aktueller Workflow: Ausdrucken der einzelnen Aufgaben, anpinnen, manuell
verschieben und parallel per Scrum-Transaktion in das SAP-System übertragen.
Dies gilt es mit aktuellen Technologien zu vereinfachen.

\section{Ziel}
Im Rahmen der Bachelorarbeit wird eine geräteübergreifende App entwickelt, die
das Scrum-Board visualisiert und den Zugriff auf Projektdaten schneller und
einfacher gestaltet.
Während der Entwicklung sollen aktuelle Technologien und Tools zum Einsatz
kommen. Kombiniert mit einem modernen Vorgehen werden die Themen Sicherheit und
Zuverlässigkeit besonders betrachtet.

In Zukunft sollen die Projektaufgaben mit Zusatzinformationen auf einem Smartphone oder
Tablet angezeigt und bearbeitet werden. Ein typischer Vorgang in dieser App ist
beispielsweise die Statusänderung von Aufgaben -- Diese kann per Drag and
Drop deutlich schneller erledigt werden.

Der bedeutendste Vorteil ergibt sich aus der ständigen Verfügbarkeit des
Projektstatus:
Das Scrum-Board muss nicht mehr physisch vorhanden sein, ein Blick in die App
genügt.
Der umständliche Zugriff über die alte, sehr umfangreiche SAP-Transaktion ist nur noch selten notwendig.

\section{SAPs Open-Source-Initiative als Chance}
Historische Entwicklungen auf Basis der SAP-eigenen Programmiersprache ABAP
(Advanced Business Application Programming) lassen sich nur schwer in einem
CI-Prozess automatisieren: Die entsprechenden Werkzeuge z.\,B. zum
ABAP-Unit-Test \cite{Majer2009} oder zur Testfallerstellung (eCATT) liegen vor,
lassen sich aber nur schwer einbinden. Eine weitere Rolle spielt die
grundsätzliche ABAP-Entwicklung im System selbst -- Eine CI-Interaktion von
außen gestaltet sich schwierig.

Deutlich mehr Möglichkeiten ergeben sich bei SAP-Entwicklungen auf Java-Basis:
Hier steht die SAP NetWeaver Development Infrastructure (NWDI) zur Verfügung
\cite{Chan2011}.
Alle CI-relevanten Tasks sind vorhanden und lassen sich automatisieren.
NWDI ist allerdings proprietär und eignet sich nicht zur Entwicklung mit anderen
Plattformen als Java EE.

Eine komplette Kehrtwende ergibt sich nun nach der Einführung von SAPUI5 als
neue SAP-Mobilplattform: Sie ist auch ohne SAP-Backend lauffähig und basiert auf
dem weit verbreiteten JavaScript-Framework jQuery \cite{Antolovic2014}.

Zusätzlich sind große Teile des Quellcodes unter dem Namen OpenUI5 auf github
veröffentlicht worden -- inklusive Buildkonfiguration und ausführlicher
Dokumentation \cite{SAP2014_1}. Sie geben einen interessanten Einblick in den
SAP-internen Workflow. 

In der Bachelorarbeit wird untersucht, welche Teile für
eigene Projekte übernommen werden können. Wo sind größere Anpassungen und
Erweiterungen notwendig? Testbarkeit spielte bei der Entwicklung des Frameworks offenbar eine größere
Rolle als früher: Frei verfügbar sind bereits der QUnit-Aufsatz OPA5 (One-Page
Acceptance tests for UI5) und ein MockServer zur Emulation von OData-Services
\cite{BoennenDreesFischerHeinzStrothmann2014}.

Durch folgende Aspekte ergeben sich gute Voraussetzungen für eine SAPUI5-CI-Toolchain mit Jenkins
(siehe \nameref{fig:CI-Toolchain}):
\begin{enumerate}
	\item Bewährte Basistechnologien
	\item Open-Source-Vorstoß
	\item Testorientierung
	\item Wachsende Community
\end{enumerate}

\section{Erkenntnisinteresse}
Besonders hervorzuheben ist die Kombination der verschiedenen Aspekte und
Vorgehen:

\begin{enumerate}
	\item Entwicklung einer aktuellen SAPUI5-App für ein bereits vorhandenes
	Altsystem auf ABAP-Basis.
	\item Die Integration des neuen NetWeaver Gateways und der entsprechenden
	OData-Services.
	\item Nutzung des Frameworks für Logon- und	Offline-Funktionen
	\item Erstellung von Testfällen anhand der Spezifikation.
	\item Zuverlässigkeit vorhandener Features nach Updates durch automatische
	Regressionstests.
	\item Aufbau der Open-Source-CI-Toolchain für eine SAP-UI5-Entwicklung
	\item Kontinuierliche Bereitstellung neuer App-Versionen für
	verschiedene Gerätetypen.
\end{enumerate}

Für alle folgenden Projekte werden diese Aspekte essentiell sein: Es gilt eine
entsprechende Toolchain zu erproben und zu etablieren, um
Softwarequalität und Erfüllung der Spezifikation nachhaltig zu gewährleisten.

\begin{figure}
	\centering
	\includegraphics[trim = 45mm 15mm 20mm 35mm, scale=0.9]{ci-toolchain}
	\caption{Angepasste CI-Toolchain.}
	\label{fig:CI-Toolchain}
\end{figure}
\chapter{Grundlagen}
\label{cha:Grundlagen}

%
% Einleitung zuletzt schreiben. Roter Faden?
% 
%Was ist testgetriebene Entwicklung? Welche Besonderheiten ergeben sich im Zusammenhang mit der Entwicklung einer JavaScript-App und der SAP-Infrastruktur?
%
%
%SAP Mobile Entwicklungen, Architekturen, Open Source, neues Backend

\autsection{Was ist testgetriebene Entwicklung?}{Mattfeld}

Beispielhaft zeigt \autoref{fig:tdd-circle} den Kern der Methode -- Die Entwicklung erfolgt In 3 Schritten:

\begin{description}
	\item[Rote Phase:] Eine noch nicht implementierte Funktion wird durch einen Test geprüft. Dieser schlägt fehl -- Ist also \textit{rot}.
	\item[Grüne Phase:] Die Funktion wird implementiert, der Testlauf ist erfolgreich und der Teststatus entsprechend \textit{grün}.
	\item[Refaktorisierung:] Der Quellcode wird verbessert, Redundanzen entfernt.
\end{description}
% Kein Absatz -> Kein Einzug
Diese Schritte wiederholen sich in einer Endlosschleife während der gesamten Entwicklungszeit. Im Unterschied zu klassischen Herangehensweisen werden zuerst Tests geschrieben und erst danach Funktionalitäten implementiert.

\begin{figure}
	\centering
	\includegraphics[width=.5\textwidth]{tdd-circle} 
	\caption[TDD-Entwicklungskreislauf]{TDD-Entwicklungskreislauf aus \cite{heise2013}.}
	\label{fig:tdd-circle}
\end{figure}

\subsection{Vorteile der Methode}
Laut Beschreibung scheint \ac{TDD} gleichbedeutend mit höherem Aufwand durch den zusätzlichen oder zumindest umfangreicheren Testumfang. Wo ist der Mehrwert? 

\ZB nach \autoref{tab:maintenance-cost} liegt der Anteil an Wartungskosten innerhalb eines Softwareprojektes bei mindestens 80\,\% -- Tendenz weiter steigend. Gemeint sind Aufwände für Erweiterungen und Fehlerbehebungen. Die Hauptaufgabe der Methode \ac{TDD} ist folglich die langfristige Reduzierung des Wartungsaufwands und damit eine Kosteneinsparung.

\begin{table}
	\centering
\begin{tabular}{ccc}
	\toprule Referenz & Zeitraum & \%\,Wartung \\ 
	\midrule Pigoski (1997) & 1985-1989 & 75\,\% \\ 
			 Frazer (1992) & 1990 & 80\,\% \\ 
			 Pigoski (1997) & 1990s & 90\,\% \\ 
	\bottomrule 
\end{tabular} 
\caption{Anteil der Wartungskosten in Softwareprojekten nach \cite[S.\,229]{PoloPiattiniRuiz2002}.}
\label{tab:maintenance-cost}
\end{table} 

\subsubsection{Keine Schnellstarts}
Klassische Entwicklungsmethoden verleiten zu Schnellstarts -- Erst programmieren, dann die Anforderungen verfeinern und nachbessern. Der Zwang erst Tests zu schreiben, führt fast zwangsläufig zu einer intensiveren Analysephase, da die zu lösenden Probleme erst verstanden und zerlegt werden müssen. Die Entwicklung wird auf die richtigen Funktionalitäten fokussiert. Daraus ergeben sich weitere Vorteile in Bezug auf Code-Umfang und -Qualität.

\subsubsection{Weniger Code -- Besserer Code}
Durch die Zerlegung in testbare Teilprobleme wird von Anfang an eine modulare Software-Architektur aufgebaut. Die Abhängigkeiten sind klar definiert, die Anwendung kann leicht erweitert oder korrigiert werden.

Im Optimalfall entsteht nur Code, der auch getestet wird. Jede Funktion muss ein vorher festgelegtes Teilproblem lösen und einen Test erfüllen. Dieses Vorgehen verhindert die Entstehung von unnötigem Code und ermöglicht die Konzentration auf das Wesentliche. Die Refactoring-Phase sorgt für die Eliminierung von Code-Duplikaten. Jede Aufgabe wird nur einmal gelöst.

\subsubsection{Jeder Fehler nur einmal}
Die stetige Überarbeitung sorgt für geringe \textit{technische Schulden} im Code \cite[S.\ 13]{Springer2015}. Weiterführend ändert sich auch der Umgang mit Bugs: Für jeden festgestellten Fehler wird ein eigener Testfall geschrieben. Erst danach beginnt die Korrektur im Code. Hierdurch werden automatische Regressionstests möglich -- Jeder Fehler wird nur einmal begangen.

All dies führt zu einer wartungsfreundlicheren Applikation, die während ihres Lebenszyklus, trotz höherem Initialaufwands, im Vergleich Kosten einsparen kann. Was ist nötig um \ac{TDD} erfolgreich einzusetzen?

\subsection{Wege zum Erfolg}
\ac{TDD} erfordert viel Umgewöhnung -- Einen Test für nicht-existierende Funktionalitäten zu schreiben erscheint ungewöhnlich. Um die Methode trotzdem zum Erfolg zu führen helfen:

\begin{itemize}
	\item Hochintegrierte Entwicklungsumgebungen
	\item Automatisierte Tests
	\item \ac{CI}
	\item Transparenz
	\item Übung und Schulungen
\end{itemize}
%
Im Folgenden die wichtigsten Richtlinien, die wir für dieses Projekt aus \ac{TDD}-Best-Practices \cite[S.\ 99]{Seidl2012} übernommen haben.

\subsubsection{Unterstützende Entwicklungsumgebung}
Auch wenn einigen Programmierern  ein einfacher Texteditor und die Kommandozeile genügen (zum Beispiel \cite{Roden2015}) -- Wichtig sind möglichst wenig Kontextwechsel während der Entwicklung, und damit eine hochintegrierte Entwicklungsumgebung. Das Werkzeug soll in den Hintergrund rücken und \ac{TDD} aktiv unterstützen.

Konkret benötigen wir Syntax-Hervorhebung der genutzten Sprache (JavaScript), die Integration von statischen Code-Analysen, Unit-Tests, Test-Treibern, Build-Systemen, Versionsverwaltungen, Verwaltung von Abhängigkeiten und einigen weiteren Funktionen, die wir in \autoref{sec:auswahl_ide} erläutern.

\subsubsection{Testfallanforderungen}
Die Codequalität eines Tests muss grundsätzlich den gleichen Ansprüchen genügen, wie die des Hauptcodes. Auch die Tests werden schließlich während des Softwarelebenzykluses immer wieder gewartet und erweitert. Lesbarkeit und Modularität sind entscheidende Erfolgsfaktoren. Im Optimalfall sind die Tests sprechend und selbsterklärend formuliert.

Eine Folge der Modularität ist Unabhängigkeit. Ein Testfall sollte nicht vom Erfolg eines anderen Testfalls abhängen, sondern allein lauffähig sein. Während eines Testlaufs würden sonst nach einem Fehlschlag auch alle weiteren Tests abbrechen. Ohne Testfallunabhängigkeit ist keine Prüfung einzelner Funktionen möglich.

Dies führt zu einem weiteren Problem, dem Ressourcenverbrauch. Um dem Entwickler zeitnah Rückmeldung zu geben, müssen Tests so schnell wie möglich abgearbeitet werden.

Wichtigste Regel: Nur ein Testfall pro Test. So kann einer Fehlerwirkung direkt eine Fehlerursache zugeordnet werden. Die Fehleranalyse bleibt kurz und effizient.

% Quelle? ISTQB?

\subsubsection{Automatisiere und sprich darüber}
Über eine nachgelagerte Test-Abteilung ist die direkte Rückmeldung an den Entwickler nicht möglich. Stattdessen sorgt ein \ac{CI}-Server für die automatische Ausführung von statischen und dynamischen Tests. Nach dem Einchecken in das Haupt-Repository sind alle Code-Bestandteile verschiedener Entwickler auf der gleichen Grundlage getestet. Veränderungen können grundsätzlich von jedem Entwickler an jeder Stelle nachvollziehbar durchgeführt werden.

Metriken bilden eine wichtige Grundlage für die Entwicklung im Team. Sie beschreiben \zB die Abdeckung des Codes durch Unit-Tests, Komplexität/Wartbarkeit und andere Ergebnisse der statischen Code-Analyse. Die, hoffentlich positive, Entwicklung der Qualität im Projekt ist so öffentlich verfügbar. 

Als Zielvorgaben sind Metriken jedoch ungeeignet. Dies würde zu engen Anpassungen an die Applikation führen, um die geforderte Abdeckung zu erreichen -- Die Tests wären schlecht wartbar und gefährden sogar den Erfolg der Methode \ac{TDD}.

\subsubsection{Die Mischung macht's}
%http://en.wikipedia.org/wiki/Test-driven_development#Best_practices
Testautomatisierung allein löst nicht alle Probleme: Nicht alles ist test- oder automatisierbar. Eine Teststrategie ist unabdingbar \cite[S.\ 33]{Spillner2014}. Um die richtigen Tests zu schreiben, bieten sich Referenzen an. Neben der Soft\-ware\-spezi\-fi\-kation existieren diverse Testorakel für bestimmte Plattformen wie iOS und Android. Wichtig ist auch Testerfahrung, \zB im Umgang mit langen Zeichenketten und Sonderzeichen.

Diese Erfahrungen unterstützen besonders im Zusammenhang mit explorativem Testen. Ergänzend unterstützen weitere Techniken wie \ac{ATDD} und \ac{BDD}. Kunde und Entwickler entwickeln gemeinsam Akzeptanzkriterien und entsprechende Beispiele. Die Sprache orientiert sich an der Geschäftslogik. Implementierungsdetails wie das Aussehen der grafischen Oberfläche finden hier noch keine Beachtung. Auf Basis der Beispiele werden Tests entwickelt, die auch vom Kunden gelesen werden können.{ }\ac{TDD} stellt sicher, dass der Code richtig geschrieben wurde, während ATDD dafür sorgt, dass es auch der richtige Code ist.

Letztlich ist der Projekterfolg mit \ac{TDD} abhängig vom Rückhalt bei Kunde und Management. Der Zwiespalt zwischen Zeit, Qualität und Kosten ist schwierig zu vermitteln. Zumal \ac{TDD} und davon abgeleitete Techniken ihre Stärken nicht bei Prototypen, sondern hauptsächlich durch Verringerung des Wartungsaufwands in längeren Projekten ausspielen.

\subsection{TDD mit JavaScript}
\label{sec:tddjs}
Grundsätzlich ist die testgetriebene Herangehensweise unabhängig von Programmiersprachen nutzbar. So existieren für alle bekannten Sprachen Unit-Test-Frameworks:  JUnit für Java, PHPUnit für PHP \usw

\subsubsection{Test-Tool-Vielfalt}
JavaScript ist anders: Es gibt kein Standard-Test-Framework. Verbreitet sind Eigenentwicklungen der größeren JavaScript-Frameworks wie QUnit für jQuery, oder unabhängige Bibliotheken wie Jasmine.

Unterscheiden lassen sich zumindest client- und serverseitige Test-Frame\-works. Die beiden bereits genannten erfordern eine HTML-Infrastruktur, in der sie die Tests im Browser ausführen. Interessant für die Nutzung im \ac{CI}-System sind Bibliotheken wie JSTestDriver oder Karma, die über eine eigene Serverkomponente mehrere Browser gleichzeitig ansprechen und automatisiert ganze Test-Suiten ausführen.

Ein ähnliches Bild gibt es bei der Vielfalt der JavaScript-Tools zur statischen Code-Analyse. Zur Auswahl stehen mindestens JSLint, JSHint und ESLint. Diese unterscheiden sich lediglich in den Einstellmöglichkeiten und reichen von strengen, nach Douglas Crockford (JSLint), bis zu komplett individuellen Regeln (ESLint).

Für einige Tests ist es nötig, nicht vorhandene Komponenten zu simulieren. Ein Beispiel könnte das, zu Entwicklungsbeginn nicht vorhandene, OData-Backend sein. Man spricht von sogenannten Mock-Objekten, die entsprechende Methodenaufrufe mit Testdaten beantworten. Für einen tieferen Einblick in Methoden sorgt ein Spy: Er protokolliert generell Methodenaufrufe, Parameter und Rückgabewerte -- In JavaScript oft Callbacks. Mit Sinon.JS existiert ein Standard-Framework für diese Funktionalitäten.

Spezielle Herausforderungen in JavaScript sind die Abhängigkeit vom \ac{DOM} und Beachtung von asynchronen Zugriffen im Rahmen von \ac{AJAX}. Beide lassen sich lösen: HTML-Fixtures (Templates) und Promises vereinfachen den Umgang. Letztere sind Bestandteile aller größeren Frameworks, \zB in Form des jQuery-Deferred-Objekts. Es fungiert als Proxy für Callbacks und verwaltet die Abhängigkeiten asynchroner Aufrufe.

\begin{figure}
	\centering
	\includegraphics[width=.95\textwidth]{ci-toolchain}
	\caption[Angepasste CI-Toolchain]{Angepasste CI-Toolchain.}
	\label{fig:CI-Toolchain}
\end{figure}

\subsubsection{Projektverwaltung}
Allein die Anzahl der bisher erwähnten JavaScript-Test-Frameworks ist hoch. Für einen kompletten Build-Vorgang fehlen zurzeit sogar noch Tools. Wie in \autoref{fig:CI-Toolchain} gezeigt, muss der Code nach dem Test noch bereinigt, komprimiert und veröffentlicht werden. Diese Aufgaben übernimmt ein JavaScript TaskRunner wie Grunt oder Gulp mit wiederum eigenen Plugins.

Zwischen den genannten Tools und Frameworks bestehen Abhängigkeiten, die ein Paketmanager auflösen sollte \cite{Fain2014}. Im Fall der Entwicklungs- und Testtools übernimmt der node.js-Paketmanager npm diese Aufgabe. Für Frontend-Frameworks wie jQuery oder SAPUI5 steht Bower zur Verfügung.

\subsubsection{Viele kleine Helfer}
Die Übersicht zu \ac{TDD} mit JavaScript zeigt: Es gibt eine große Vielfalt an Frameworks. Diese sind oft klein und erfüllen nur spezielle Aufgaben. Die Mehrheit wird in OpenSource-Communities entwickelt und beeinflusst sich gegenseitig stark. Das Entwicklungstempo ist hoch, und so sind bestehende Best Practices innerhalb weniger Monate oder Jahre hinfällig.

Die Herausforderung während der Implementierung wird in der Auswahl der geeigneten Tools und Frameworks liegen: Welche Teile können aus SAPUI5 übernommen werden, was muss durch externe Module ergänzt werden? Ist die große Modularität des Java-Script-Workflows eine Stärke aus der ein Standard entwickelt werden kann?

\autsection{SAP Mobile}{Mattfeld}
Das UI5 Development Toolkit for HTML5 (SAPUI5) steht im Mittelpunkt der aktuellen SAP-Mobile-Strategie. Als geräteübergreifendes HTML5-Frame\-work soll es schrittweise die SAP GUI und mobile Eigenentwicklungen ablösen. Viele Standardanwendungsfälle werden bereits von den mitgelieferten Fiori Apps auf UI5-Basis abgedeckt. Die ebenfalls neuen Screen Personas sind dagegen nur eine Übergangslösung, um vorhandene Transaktionen vereinfacht darzustellen.

Verfügbar ist UI5 als proprietäres SAPUI5 für SAP-Kunden oder als quelloffenes OpenUI5. Da der Kern beider Versionen identisch ist, sprechen wir im Folgenden nur noch von UI5.

Laut Gartner ist die zugehörige SAP Mobile Platform (SMP) schon jetzt führende Entwicklungs-Plattform für mobile Applikationen \cite{Gartner2014}. Auch die Ankündigung der SAP, den Mainstream Support der SMP 3.0 bis 2020 zu verlängern \cite{Mielke2015}, unterstreicht den Zukunftscharakter dieser Technologie. Bisherige Lessons Learned aus UI5-Projekten zeigen vor allem Potenzial in den Bereichen Entwicklungsumgebung, Test-Entwicklung und -Automatisierung \cite{Thiebes2014}, die wir daher besonders betrachten.

\subsection{UI5-Interna}
Das Framework besteht aus mehreren Bibliotheken: Immer geladen werden \textit{sap.ui.core} und \textit{sap.ui.commons}, das Standard-Bedienelemente beinhaltet. Für unsere App werden wir außerdem \textit{sap.m} nutzen, das seine UI-Elemente dynamisch an die Bildschirmgröße anpasst. Während diese Bibliotheken auch im quelloffenen OpenUI5 vorhanden sind, haben nur zahlende Kunden Zugriff auf \textit{sap.viz} und \textit{sap.makit} zur Diagrammerstellung.

Die Entwicklung von SAPUI5 begann schon 2011, ist jedoch keine vollständige Neuentwicklung. Integriert werden verschiedene Open-Source-Framworks wie jQuery, jQuery Mobile, data.js, QUnit und Sinon.JS -- Kombiniert ergeben sie ein JavaScript-Framework, dass sich auch für umfangreiche Unternehmensanwendungen eignet und durch jQuery-Plugins oder eigene Steuerelemente erweitert werden kann.

Im Unterschied zu jQuery Mobile als Einzelframework unterstützt es Modularisierung, Internationalisierung, Routing, Testbarkeit und vieles mehr. Es befindet sich damit auf einer Ebene mit anderen Enterprise Frameworks wie Sencha Ext JS inkl. Sencha Touch oder AngularJS, die ähnliche Ansätze verfolgen.

UI5-Apps basieren auf der \ac{MVC}-Architektur. Die Anwendungslogik wird in den Browser verlagert. Der zuständige Controller ist grundsätzlich in JavaScript implementiert. Views werden dynamisch ebenfalls per JavaScript erzeugt oder statisch als XML, HTML oder JSON definiert. Das Model bezieht die App über \ac{REST}-Services. 


\subsection{Architekturen}
\label{sec:architekturen}
Das SAP Gateway liefert entsprechende OData-Services nach dem REST-Prinzip:

\begin{itemize}
	\item Zustandslos -- Operationen sind in sich abgeschlossen
	\item Übertragung im XML-Format
	\item Navigationseigenschaften -- \zB Verlinkung vom Kunden zum Projekt
	\item URL-adressierbare Objekte -- \zB \textit{http://\dots/Kunde(3)/Projekt(1)}
	\item Operationsbasiert -- \zB \textit{GET, POST, PUT, DELETE, \dots}
\end{itemize}
Diese Services lassen sich auf Basis verschiedener Quellen implementieren und sind insgesamt leichter handhabbar als ihre Vorgänger SOAP/WSDL. \autoref{sec:neues_backend} zeigt die wichtigsten Möglichkeiten.

Die für mobile Geschäftsanwendungen oft geforderte Offline-Fähigkeit ergibt sich allerdings erst im Zusammenspiel mit der \textit{SAP Mobile Platform 3.0} oder ihrem Cloud-Pendant \textit{HANA Cloud Platform mobile services} auf Basis der ehemaligen Sybase Unwired Platform. Single Sign-on (SSO), Push-Benachrichtigungen und Fernzugriff ohne VPN sind weitere Zusatzfunktionen beider Lösungen. 

Daraus folgen mindestens fünf Deployment- und App-Architekturen, die sich vor allem in den nutzbaren Business- und Geräte-Funktionen unterscheiden:
\begin{enumerate}
	\item Web
	\item Fiori-Launchpad-Integration
	\item Zugriff über Fiori Client
	\item Hybrid-App
	\item Angepasster Fiori Client
\end{enumerate}
Methode 1 eignet sich für alle Arten von UI5-Apps. Die Veröffentlichung kann auf jedem beliebigen Webserver erfolgen. Ein SAP-Bezug ist nicht nötig. Allerdings werden keine weiteren Business- oder Geräte-Funktionen unterstützt.

\begin{figure}[b]
	\centering
	\includegraphics[width=.85\textwidth]{SAPUI5_Hybrid_Fiori2} 
	\caption[UI5-Eigenentwicklung im Fiori Launchpad]{UI5-Eigenentwicklung im Fiori Launchpad.}
	\label{fig:SAPUI5_Hybrid_Fiori2}
\end{figure}

Zur Fiori-Launchpad-Integration ist eine bestimmte App-Struktur und Modularisierung nötig. Die App wird normalerweise über ein ABAP-Reposit\-ory veröffentlicht. Vorteile sind die SSO-Möglichkeit und ein einheitliches Benutzererlebnis durch die Integration in Fiori wie in \autoref{fig:SAPUI5_Hybrid_Fiori2}.

Der Fiori Client ist für iOS und Android verfügbar. Wird das Launchpad durch ihn aufgerufen, speichert er Zugangsdaten und App-Inhalte im Cache: Die Performance wird verbessert. Weitere Gerätefunktionen sind zur Zeit nicht nutzbar.

UI5-Apps lassen sich beliebig in eigene Hybrid-Apps auf Cordova-Basis integrieren, um native Gerätefunktionen wie Kamera oder GPS aufzurufen. Geschäftsfunktionen und der Zugriff auf die SAP-Infrastruktur gelingt aber nur mit einem angepassten Fiori Client. Der Client-Quellcode ist als Teil des Mobile Platform SDKs verfügbar. Zusammen mit den sogenannten Kapsel Plugins (Cordova) ist das der einzige Weg, um die Offline- und Push-Funktionen der Mobile Platform (SMP) nutzbar zu machen.

\subsection{Open-Source-Initiative als Chance}
\label{sec:os_chance}
Historische Entwicklungen auf Basis der SAP-eigenen Programmiersprache ABAP
(Advanced Business Application Programming) lassen sich nur schwer in einem
CI-Prozess automatisieren: Die entsprechenden Werkzeuge z.\,B. zum
ABAP-Unit-Test \cite{Majer2009} oder zur Testfallerstellung (eCATT) liegen vor,
lassen sich aber nur schwer einbinden. Eine weitere Rolle spielt die
grundsätzliche ABAP-Entwicklung im System selbst -- Eine CI-Interaktion von
außen gestaltet sich schwierig.

Deutlich mehr Möglichkeiten ergeben sich bei SAP-Entwicklungen auf Java-Basis:
Hier steht die SAP NetWeaver Development Infrastructure (NWDI) zur Verfügung
\cite{Chan2011}.
Alle CI-relevanten Tasks sind vorhanden und lassen sich automatisieren.
NWDI ist allerdings proprietär und eignet sich nicht zur Entwicklung mit anderen
Plattformen als Java EE.

Eine komplette Kehrtwende ergibt sich nun nach der Einführung von SAPUI5 als
neue SAP-Mobilplattform: Sie ist auch ohne SAP-Backend lauffähig und basiert auf
dem weit verbreiteten JavaScript-Framework jQuery \cite{Antolovic2014}.
Zusätzlich sind große Teile des Quellcodes unter dem Namen OpenUI5 auf GitHub
veröffentlicht worden -- inklusive Buildkonfiguration und ausführlicher
Dokumentation \cite{SAP2014_1}. Sie geben einen guten Einblick in den
SAP-internen Workflow. 

In der Bachelorarbeit wird untersucht, welche Teile für
eigene Projekte übernommen werden können. Wo sind größere Anpassungen und
Erweiterungen notwendig? Testbarkeit spielte bei der Entwicklung des Frameworks offenbar eine größere
Rolle als früher: Frei verfügbar sind bereits der QUnit-Aufsatz OPA5 (One-Page
Acceptance tests for UI5) und ein Mock-Server zur Emulation von OData-Services
\cite{BoennenDreesFischerHeinzStrothmann2014}.

\subsubsection{Gute Voraussetzungen} 
Folgende Aspekte begünstigen den Aufbau einer SAPUI5-CI-Toolchain mit Jenkins (siehe \nameref{fig:CI-Toolchain}):
\begin{quote}
	\begin{enumerate}
		\item Bewährte Basistechnologien
		\item Open-Source-Vorstoß
		\item Testorientierung
		\item Wachsende Community
	\end{enumerate}
\end{quote}


\autsection{Neues Backend: SAP Gateway}{Azimi}
\label{sec:neues_backend}
Das SAP Gateway wurde im Mai 2011 auf den Markt gebracht, um die Reichweite von SAP-Geschäftsanwendungen zu erhöhen. 
Dazu bietet das Gateway eine offene, REST-basierte Schnittstelle, um einen einfachen Zugriff auf SAP-Systeme zu ermöglichen, siehe \autoref{fig:GatewayArchitektur}.
Das offene Protokoll OData wird verwendet, da es viel genutzt, bekannt und leicht zu lernen ist. Damit soll die SAP-Welt jedem Entwickler zugänglich sein, der OData versteht, da kein spezielles SAP-Fachwissen mehr nötig ist, um eine Oberfläche für Anwendungen und den Zugriff auf SAP-Daten zu entwickeln.

Durch die Bereitstellung der OData-Schnittstelle müssen der Backend-Entwickler und der Frontend-Entwickler nicht wissen, was der jeweils andere im Detail programmiert. Eine klare Definition der benötigten Schnittstellen ist allerdings zwingend notwendig und erleichtert die gesamte Entwicklung \cite[S.\ 31-45]{BoennenDreesFischerHeinzStrothmann2014}.

\begin{figure}
	\centering
	\includegraphics[width=.95\textwidth]{Gateway_architektur_page45} 
	\caption[SAP-Gateway-Integrationsmodell]{SAP-Gateway-Integrationsmodell aus \cite[S.\ 45]{BoennenDreesFischerHeinzStrothmann2014}.}
	\label{fig:GatewayArchitektur}
\end{figure}

\subsection{Best Practices}
Um möglichst hochwertige Anwendungen mit Hilfe von OData-Services zu entwickeln, empfiehlt SAP folgende Best Practices zu beachten\cite[S.\ 561-563]{BoennenDreesFischerHeinzStrothmann2014}: 
\begin{itemize} 
	\item Kommunikation zwischen den Entwicklungsteams zur genauen Spezifikation der Schnittstellen.
	\item Frühzeitiger Entwicklungsbeginn des OData-Services, damit dem Front\-end-Entwicklerteam möglichst früh ein konsumierbarer Service zur Verfügung steht. 
	\item Gruppierung der Services in Module, sodass die Anwendung auf bereits vollständig entwickelte Teil-Services zugreifen kann. 
	\item Gründliches Testen der vollständig entwickelten OData-Services mit Hilfe von Test-Tools, sowie zusätzliche Tests mit der entwickelten Anwendung. 
	\item Frühzeitiges Testen in der Produktivumgebung.
	\item Nutzen der Query-Option \textit{sap-ds-debug=true} im Browser.
 \end{itemize}

\subsection{Deployment-Optionen}
Für das Deployment des SAP Gateways stehen verschiedene Optionen zur Verfügung. Diese werden nachfolgend beschrieben.

\subsubsection{Eingebettetes Deployment}

Bei dieser Variante des Deployments werden alle Komponenten des SAP Gateways auf dem SAP-Backend-System installiert. Ab SAP NetWeaver\,7.40 ist das Gateway Bestandteil der Standardinstallation. Hauptvorteil ist die Reduzierung der Laufzeit um einen Remote Call: Der Service wird direkt auf dem Backend registriert. Der Nachteil an dieser Variante ist, dass das SAP Gateway für jedes Backend separat konfiguriert werden muss.

Zudem rät SAP davon ab, SAP-Backend-Systeme mit eingebettetem Gateway als Hub-System für andere Backends zu verwenden. Abhängigkeiten im Aktualisierungsprozess können ein Hub-Update bei bereits aktualisiertem Backend-System verhindern \cite[S.\ 52-54]{BoennenDreesFischerHeinzStrothmann2014}.

\subsubsection{Hub-Deployment mit Entwicklung auf dem SAP-Backend}

Die Gateway-Kernkomponenten werden auf einem zusätzlichen SAP-Gate\-way-Hub-System installiert. Zusätzlich muss auf dem SAP-Backend-System die Business-Enablement-Provisioning-Komponente (\textit{IW\_BEP}) installiert werden. Bei dieser Variante wird der Service auf dem Backend-System entwickelt und veröffentlicht (deployt). Der Service wird anschließend auf dem Gateway Hub registriert. Beim Hub-Deployment ist es möglich, mehrere SAP-Backends mit dem Gateway zu verbinden. Zudem gibt es hier keinen direkten Zugriff auf das SAP-Backend-System \cite[S.\ 54]{BoennenDreesFischerHeinzStrothmann2014}.

\subsubsection{Hub-Deployment mit Entwicklung auf dem Hub}

Alle Komponenten des SAP Gateways werden auf dem Hub-System installiert. Diese Variante wird beispielsweise bei eingeschränktem Zugriff auf das Backend-System eingesetzt, falls eine Entwicklung auf diesem nicht möglich ist. Zwingend ist diese Variante bei Release-Ständen unter SAP NetWeaver\,7.40, dort ist eine Installation von \textit{IW\_BEP}  auf dem Backend nicht erlaubt. 

Die Entwicklung des Services findet auf dem Hub-System statt. Vorteil: Das SAP-Backend-System wird nicht modifiziert. Allerdings muss bei der Entwicklung auf eine bereits vorhandene Schnittstelle wie \zB einen \ac{RFC} oder ein \ac{BAPI} zurückgegriffen werden, die das Backend bereitstellt \cite[S.\ 55]{BoennenDreesFischerHeinzStrothmann2014}.


\subsection{OData und REST}
Das \ac{OData} ist ein REST-basiertes Datenzugriffsprotokoll, welches unter der \ac{OSP} herausgegeben wurde und auf verbreiteten Standards wie Atom Pub, XML und JSON aufbaut \cite[S.\ 69-70]{BoennenDreesFischerHeinzStrothmann2014}.
Um eine \ac{REST}-konforme Architektur einzuhalten müssen folgende Eigenschaften erfüllt werden \cite[S.\ 65-66]{BoennenDreesFischerHeinzStrothmann2014}:

	\begin{enumerate}
	\item \textbf{Client-Server-Architektur} --
	Client und Server werden von einer einheitlichen Schnittstelle getrennt.
	\item \textbf{Zustandslosigkeit} --
	Es werden keine Daten zu den einzelnen Abfragen gespeichert. Daher muss jede Abfrage alle benötigten Informationen beinhalten.
	\item \textbf{Pufferbarkeit} --
	Die Antworten des Servers müssen spezifizieren, wie lange Daten gespeichert werden dürfen, damit keine veralteten oder falschen Daten verwendet werden.
	\item \textbf{Mehrschichtigkeit} --
	Der Client weiß nicht, ob er direkt mit dem Backend oder nur mit einem Server in der Mitte verbunden ist.
	\item \textbf{Einheitliche Schnittstelle} -- Die Ressourcen werden nach einem einheitlichen Verfahren identifiziert, wie \zB Name, Pfad oder Primärschlüssel. Abfrage und Manipulation von Ressourcen geschehen nach einem einheitlichen Verfahren.
	\end{enumerate}
Durch OData werden dem Konsumenten definierte Schnittstellen für verschiedene Datenquellen zur Verfügung gestellt und soll so die Zusammenarbeit zwischen verschiedenen Plattformen ermöglichen \cite[S.\ 69-70]{BoennenDreesFischerHeinzStrothmann2014}.

\subsubsection{Datenmodell}
Für den OData-Service muss ein \ac{EDM} definiert werden, welches über das Metadaten-Dokument für Clients zum Abruf  bereit steht. Im \ac{EDM} werden die einzelnen Entitäten, Entitätstypen und ihre Attribute definiert. Des Weiteren werden die Kardinalitäten zwischen den einzelnen Entitäten über Assoziationen festgelegt, sodass über Navigationsattribute zwischen den einzelnen Entitäten navigiert werden kann \cite[S.\ 72-73]{BoennenDreesFischerHeinzStrothmann2014}.


\subsubsection{Servicedokumente}

Im Servicedokument (siehe Listing \ref{lst:Servicedokument}) werden alle Entitätsmengen, welchen immer ein Entitätstyp zugrunde liegt, eines Services angezeigt. Zudem werden SAP-spezifische Metainformationen angezeigt, welche unter anderem beschreiben, ob die Entitätsmengen das Hinzufügen, Ändern oder Entfernen von Einträgen oder das erlauben. Erreichbar ist das Servicedokument über den Aufruf des Services ohne zusätzliche Parameter \cite[S.\ 77-81]{BoennenDreesFischerHeinzStrothmann2014}. Die URI zu einem Servicedokument könnte Beispielsweise so aussehen: \\\textit{http://host:port/sap/opu/odata/sap/Z\_ZAV\_SCRUM\_SRV/}\\

\begin{listing}[h]
	\inputminted{xml}{src/servicedokument-short.xml}
	\caption{Servicedokument Z\_ZAV\_SCRUM\_SRV}
	\label{lst:Servicedokument}
\end{listing}

Im Service-Metadokument (siehe Listing \ref{lst:Metadata}) sind die gesamten Metainformationen, inklusive Datenmodell, Entitätstypen, Entitätsmengen, Assoziationen und Navigationseigenschaften des Services enthalten. Dadurch kennt der Zugreifende alle Informationen über Struktur, Ressourcen und Verknüpfungen \cite[S.\ 81-82]{BoennenDreesFischerHeinzStrothmann2014}. Aufgerufen wird das Service-Metadokument über den zusätzlichen Parameter \textit{\$metadata} wie im folgendem Beispiel: 
\\\textit{http://host:port/sap/opu/odata/sap/Z\_ZAV\_SCRUM\_SRV/\$metadata}

\begin{listing}[h]
	\inputminted{xml}{src/metadata.xml}
	\caption{Metadaten Z\_ZAV\_SCRUM\_SRV}
	\label{lst:Metadata}
\end{listing}
\chapter{Anforderungsanalyse}
\label{cha:Anforderungsanalyse}

\autsection{App-Spezifikation}{Mattfeld}
\label{sec:app-spec}
Kurzbeschreibung der App:
\begin{itemize}
\item Darstellung eines Scrum-Boards mit mehreren Spalten
\item Darstellung der Aufgaben in den zugehörigen Spalten
\item Möglichkeit für den User, Aufgaben zwischen den Spalten hin- und herzuschieben
\item Update des Aufgabenstatus  in der Datenbank durch Verschieben
\end{itemize}
\SuperPar
Relevante Backend-Tabellen:
\begin{itemize}
\item zav\_kunden: Kundentabelle
\item zav\_projekte: Projekte je Kunde
\item zav\_sprints: Sprints je Projekt je Kunde
\item zav\_aufgaben: Aufgaben je Projekt, je Kunde, teilweise Sprints zugeordnet
\item zav\_formcust: Definition von Spalten je Projekt je Kunde; Zuordnung von Aufgabenstatus zu Spalten
\end{itemize}


\subsection{Anforderungen}
Die Startsequenz der App ist wie folgt:
\begin{enumerate}
	\item Anmeldung am System
	\item Auswahl eines Kunden
	\item Auswahl eines (zum Kunden gehörigen) Projektes
	\item Auswahl eines (zum Projekt gehörigen) Sprints
	\item Darstellung eines Scrum-Boards mit mehreren Spalten
	\item Darstellung der zu den Spalten gehörigen Aufgaben als Aufgabenzettel
\end{enumerate}
Die Spalten-Überschriften sind dabei dynamisch, jedoch ergibt sich für jedes Projekt ein jeweils eindeutiger Satz an Spalten. Die Einordnung von Aufgaben in den Spalten ergibt sich aus dem Status der Aufgabe; Die Zuordnung von Status-Konstellationen zu Spalten folgt der Tabelle (ZAV\_FORMCUST).

\begin{figure}
	\centering
	\includegraphics[width=.8\textwidth]{Architektur2} 
	\caption[App-Architektur]{App-Architektur nach \cite{openSAP2014_1}.}
	\label{fig:Architektur2}
\end{figure}

\subsection{Architektur}
\label{sec:app-architektur}
Wie im \autoref{sec:architekturen} beschrieben, gibt es viele Möglichkeiten eine UI5-App zu betreiben und zu veröffentlichen. Die Entscheidung richtet sich nach den benötigten Geräte- und Geschäftsfunktionen sowie der SAP-Integrationstiefe. 

In der ersten Version der Scrum-App sind weder Offline-Funktionen, noch native Gerätezugriffe gefordert. Einzig der Zugriff ohne Intranet-Verbindung erfordert besondere Beachtung. Die empfohlene Lösung ist die Weiterleitung des OData-Services nach außen über die Hana Cloud Platform, siehe \autoref{fig:Architektur2}. Der SAP Cloud Connector stellt die sichere Verbindung zwischen SAP-Backend vor Ort und der Cloud her. Vorteil: Das SAP-Backend des Kunden ist nach außen nicht sichtbar, die Firewall-Konfiguration muss nicht angepasst werden. App-Benutzer melden sich direkt an der Cloud-Plattform an, benötigen also keinen Intranet-Zugriff per VPN mehr.


%\subsection{Aufwandsschätzung}


\section{Testkonzept}
Das Testkonzept orientiert sich an IEEE\,829 \cite{IEEE829}, angepasst an die kleinere Projektgröße. Es ist als Dokument für alle Aktivitäten während des gesamten Projekts gültig. Darüber hinaus soll es als Vorlage für zukünftige App-Entwicklungen dienen. Ziel ist die Umsetzung des fundamentalen Testprozesses nach Spillner \cite{SpillnerRossnerWinterLinz2014}. In anderen Teilen der Arbeit behandelte Testwerkzeuge werden nicht beschrieben. Berücksichtigt werden folgende Dokumente:
\begin{itemize}
	\item Anforderungsspezifikation des Kunden, siehe \autoref{sec:app-spec}
	\item UI5 Development Conventions and Guidelines \cite{SAP2014_1}, insbesondere:
	\begin{itemize}
		\item JavaScript Coding Guidelines
		\item UI5 Control Development Guidelines
		\item Product Standards/Acceptance Criteria
		\item Git Guidelines
		\item File Names and Encoding	
	\end{itemize}
	\item SAP Gateway Best Practices (\cite[S.\ 561-563]{BoennenDreesFischerHeinzStrothmann2014})
	\item Projektvorgehen: Testgetrieben inkl. \ac{CI} (siehe \autoref{cha:Grundlagen})
\end{itemize}


\subsubsection{Testobjekte}
Der Architektur aus \autoref{sec:app-architektur} folgend, besteht die Anwendung aus zwei Hauptkomponenten. Beide sind Teil des Testkonzepts:
\begin{itemize}
	\item HTML5-Frontend (OpenUI5 1.26.8)
	\item SAP-Backend
	\begin{itemize}
		\item Funktionsbausteine (ECC 6.0, EHP 7, NW 7.4)
		\item OData-Service (SAP Gateway NW 7.4)
	\end{itemize}
\end{itemize}
Nicht einzeln getestet werden die JavaScript-Frameworks wie OpenUI5 und die verwendeten Test-Frameworks. Die Testbasis der entsprechenden Open-Source-Projekte ist mit diversen Unit- und Akzeptanztests sehr umfangreich. Eine projektinterne Test-Wiederholung bringt keinen Mehrwert.


\autsubsection{Steuerung und Planung}{Mattfeld}
Durch die kleine Teamgröße sind Entwickler gleichzeitig Tester (Test-Manager, Designer, Automatisierer, \dots).
Sie erarbeiten gemeinsam mit dem Kunden nach dem Prinzip des \ac{BDD} allgemein verständliche Anforderungen. Die Beschreibung der Fallbeispiele erfolgt im Schema von automatisierbaren \textit{Wenn-Dann}-Sätzen.

Fehlerkosten verdoppeln sich in jeder Entwicklungsphase. Der Kunde wird daher besonders stark in die Designphase einbezogen, um Kosten durch späte Änderungen zu vermeiden.

\subsubsection{Risiko und Kosten}
Insgesamt sind Fehlerkosten schwer abzuschätzen. Schadenshöhe und Wahrscheinlichkeit sind durch interne, ergänzende Verwendung der App überschaubar: Die Scrum-Aufgaben werden nur angeschaut oder im Status verändert. Nutzereingaben über den Client müssen allerdings immer als kritisch eingestuft werden. Gültigkeitsprüfungen in der App dienen nur der schnellen Rückmeldung an den User und Trafficvermeidung, nicht der Sicherheit. Das Backend ist daher ausführlich zu betrachten. 

Durch eine Fehlfunktion besteht eine Gefahr für laufende Projekte bei diversen Kunden. Daraus folgen direkte Schäden und Imageverlust. Fehlerkorrekturkosten werden durch TDD und CI eingegrenzt -- Regressionstests und Deployment sind automatisiert.

Testkosten sind ebenso schwierig vorauszusehen. Vor allem der geringe Reifegrad des Entwicklungsprozesses erschwert die Schätzung: Bisherige Erfahrungen sind durch den prototypischen Projekt-Charakter nicht vorhanden. Der Testplan kommt zum ersten Mal zum Einsatz. Allerdings wird die Testbarkeit und Modularität der Software durch testgetriebene Entwicklung sichergestellt. Zusätzlich ist die Testinfrastruktur von Anfang an vorhanden und erfüllt alle Anforderungen an Continuous Integration. Die Projektteilnehmer sind mit diesen Werkzeugen vertraut und beherrschen die Grundlagen des Testens nach ISTQB.

Die höchste Priorität haben die Tests des sicherheitskritischen Backends. In der App-Komponente decken die Akzeptanzkriterien Funktionalitäten mit hoher Nutzungshäufigkeit ab. Sie stehen im Mittelpunkt der Kundenwahrnehmung des Produktes und werden entsprechend wichtig eingestuft. 

Testaufwand: Eine vollständige Analyse der Aufwände ist nicht möglich und Erfahrungswerte sind noch nicht vorhanden. Diese Herausforderungen stellen ein großes Projektrisiko dar. Zumindest durch bekannte QM-Werkzeuge und hohe Automatisierung soll die Erfolgswahrscheinlichkeit erhöht werden. Den Testaufwand schätzen wir bei diesem Prototypen mit mindestens 50\,\% des Gesamtentwicklungsaufwands ein.

\subsubsection{Qualitätsziele und Testabedeckung}
Testendkriterium ist eine Anforderungsüberdeckung von 100\,\%. Ebenso müssen die sicherheitskritischen Äquivalenzklassentests des Backends zu 100\,\%. erfolgreich laufen. Testabbruchkriterien sind fehlgeschlagene Unit- und Akzeptanztests. Fehler der statischen Analyse führen ebenso zum Abbruch, Warnungen oder Schwankungen in den Komplexitäts-Metriken allerdings nicht.

Die nicht-funktionalen Anforderungen des UI5-Design-Guides können nur teilweise durch statische Analysen automatisiert überprüft werden. Gezielte Reviews während der Designphase und bei größeren Änderungen ergänzen die automatische Analyse. Die Performance der App wird im Rahmen der Systemtests überprüft. Die Akzeptanztests werden hierzu mit kurzen Timeouts angepasst.

\subsubsection{Konfigurationsverwaltung}
Die Konfigurationsverwaltung erfolgt per Git über die Plattform GitHub. Es gelten die UI5 Guidlines zu Git Commits: Es muss jederzeit nachvollziehbar sein, wer, wann, warum, welche Codezeilen geändert hat.

Eingecheckt wird nicht nur der Programmcode, sondern auch die Dokumentation, Tests und die CI-Toolchain. Das Prinzip \textit{Infrastructure as a Code} ist vollständig umgesetzt. Das bedeutet, die gesamte Build- und Deployment-Umgebung kann auf einem neuen System automatisch wiederhergestellt werden. 

Die Abhängigkeiten der Tool- und Framework-Versionen sind in Konfigurationsdateien hinterlegt und werden über die Paketmanager aufgelöst. Über Git werden Zwischenversionen der App gekennzeichnet oder in Branches ausgegliedert. Die Wiederherstellung eines bestimmten Entwicklungsschrittes inklusive Dokumentation, Buildumgebung und Tests ist stets gewährleistet. 

Das CI-System stellt die Gültigkeit der Konfigurationsverwaltung sicher. Jede Neuerung in der Versionskontrolle löst einen vollständigen Neuaufbau der Buildumgebung auf dem CI-Server anhand der Konfiguration aus.


\subsubsection{Fehlermanagement}
Ein Testprotokoll wird für jeden Build separat auf dem CI-Server vorgehalten. Das gesamte Projektteam hat Zugriff auf das Protokoll. Zusätzlich wird der Build-Status im Repository visualisiert. Das Team ist jederzeit über den Teststatus informiert.

Durch \ac{TDD} existiert theoretisch kein ungetesteter Code. Die QA-Pipeline wird in einem \textit{Private Build} auf dem Entwicklungsrechner komplett durchlaufen. Noch nicht implementierte Funktionalitäten bzw. die zugehörigen roten Tests werden auch nicht in die globale Versionsverwaltung eingecheckt. Neue Fehlerwirkungen werden vor allem während der höheren Teststufen wie Integrations-, System- und Akzeptanztests auftreten. Denkbar sind auch Inkompatibilitäten mit Browsern und Endgeräten oder Probleme mit dem CI-Prozess selbst.

 Da alle Projektartefakte in der Versionsverwaltung eingecheckt werden, ist auch das Fehlermanagement zentral für alle Projektteilnehmer. Das Schema einer Fehlermeldung ist festgelegt und beinhaltet eine eindeutige Fehlernummer, Entdecker, Erfassung und Problembeschreibung. Angaben zur zugehörigen Anforderung, Fehlerquelle und Reproduktion werden wenn möglich eingetragen. Über verschiedene Labels werden für dieses Projekt individuell gesetzt: Fehlerklasse (Bug, Enhancement, Documentation), Priorität (open, wontfix) und Status (new, open, approved, invalid, question, duplicate, works, fixed).

Testfälle und Korrekturen werden über einen Fork eingebracht. Beides kann per Kommentarfunktion diskutiert werden. Ist der Fehler durch einen Testfall nachgewiesen und die Code-Korrektur akzeptiert, schließt ein Kommentar im Commit automatisch das Issue Ticket.

%\subsection{Design und Analyse}
%Testbasis prüfen (Anforderungen), Testbarkeit prüfen, Logische Testfälle

\subsection{Durchführung}
\label{sec:tests}
Die Testfälle unterteilen sich in verschiedene Teststufen:
\begin{itemize}
	\item Komponententests der Funktionsbausteine
	\item Integrationstests zwischen Backend und Gateway
	\item Isolierter Akzeptanztests der App mit Mockserver
	\item Systemtest mit App und Gateway
\end{itemize}

\subsubsection{Akzeptanztest-Szenario 1: Kundenübersicht}

\textbf{Given} I start the app\\
\textbf{When} I look at the screen\\
\textbf{Then} I should see the customer list\\
\textit{and} the customer list should have entries\\

\subsubsection{Akzeptanztest-Szenario 2: Projektübersicht}

\textbf{Given} I start the app\\
\textbf{When} I press on customer 1\\
\textbf{Then} I should be taken to customer 1\\
\textit{and} I should see the project list of customer 1\\
\textit{and} the project list should have entries\\



\subsubsection{Äquivalenzklassentest}
Um alle möglichen Eingabewerte für einen Funktionsbaustein zu ermitteln, werden Äquivalenzklassen für die jeweiligen Eingabeparameter gebildet, siehe \autoref{tab:aquivalenzklassen-FubaProjekte}. Hierbei ist darauf zu achten, dass für jeden Parameter auch negative Werte getestet werden. Da die Anzahl der Testfälle bei einfacher Multiplikation der möglichen Kombinationen schnell zu einer erheblichen Anzahl ansteigt, werden gültige Äquivalenzklassen in den Testfällen kombiniert. Ungültige Äquivalenzklassen dürfen wiederum nur mit gültigen Äquivalenzklassen kombiniert werden, sodass für jede ungültige Äquivalenzklasse ein eigener Testfall erstellt wird \cite[S.\ 110-115]{Spillner2010}. Die daraus entstandenen Testfälle sind in \autoref{tab:test-read-projektet} zu sehen. Damit wird eine Äquivalenzklassenüberdeckung von 100\,\% für den zu testenden Funktionsbaustein erreicht.

\begin{table}[h]
	\centering
\begin{tabular}{cccc}
	\toprule Äquivalenzklasse & Parameter & Eingabewert & ÄK gültig?\\
	\midrule ÄK1 & Kunde  & ''TKMI''  & Ja\\ 
			ÄK2 & Kunde & ''tkmi'' & Nein\\
			ÄK3 & Kunde & - & Nein\\
			\midrule
			ÄK4 & Projekt & ''TM-SCRUM'' & Ja\\
			ÄK5 & Projekt & ''tm-scrum'' & Nein\\
			ÄK6 & Projekt & - & Ja\\
			\midrule
			ÄK7 & Substring & ''TM-SCRUM'' & Ja\\
			ÄK8 & Substring & ''tm-scrum'' & Ja\\
			ÄK9 & Substring & ''asdf'' & Nein\\
			ÄK10 & Substring & - & Ja\\
			\midrule
			ÄK11 & Projekt und Substring & nicht leer & Nein\\
	\bottomrule 
\end{tabular} 
\caption{Äquivalenzklassen für den Funktionsbaustein Z\_SCRUMUI5\_READ\_PROJEKTE}
\label{tab:aquivalenzklassen-FubaProjekte}
\end{table} 

\begin{table}[h]
	\centering
\begin{tabular}{cccccc}
	\toprule Testfall & getestete ÄK & Kunde 	& Projekt 		& Substring 	&  Ergebnis \\ 
	\midrule     1 	 &  1,6,10 		& ''TKMI'' 	& - 			& - 			&  Anzeige erfolgt\\ 
			 2 	 & 4			& ''TKMI'' 	&''TM-SCRUM'' & - 			& Anzeige erfolgt\\ 
			 3	 &  7			& ''TKMI'' 	& - 			& ''TM-SCRUM'' & Anzeige erfolgt\\ 
			 4	 &  8			& ''TKMI'' 	& - 			& ''tm-scrum''	& Anzeige erfolgt\\ 
			5 	 &  2			& ''tkmi'' 	& - 			& - 			& Keine Anzeige\\ 
			6 	 &  3			& - 		& - 			& - 			& Keine Anzeige\\ 
			7 	 &  5			& ''TKMI'' 	& ''tm-scrum''	& - 			& Keine Anzeige\\ 
			8 	 &  9			& ''TKMI'' 	& - 			& ''asdf'' 		& Keine Anzeige\\ 
			 9 	 & 11			& ''TKMI'' 	& ''TM-SCRUM''	& 'TM-SCRUM''		& Keine Anzeige\\ 
	\bottomrule 
\end{tabular} 
\caption{Ermittelte Testfälle für den Funktionsbaustein Z\_SCRUMUI5\_READ\_PROJEKTE}
\label{tab:test-read-projektet}
\end{table} 

%Äquivalenzklassen-Überdeckung = (Anzahl getesteter ÄK / Gesamtanzahl ÄK) x 100\%

\subsubsection{Testfälle OData-Service}
Die relevanten Testfälle (siehe \autoref{tab:testfälle-OData}) für den OData-Service werden durch die verwendeten OData-Funktionen bestimmt. Dabei ist zwischen den allgemeinen Testfällen für den kompletten OData-Service, sowie den Funktionen die für jede Entität einzeln getestet werden zu unterscheiden.

\begin{table}[h]
	\centering
\begin{tabular}{ccccc}
	\toprule Testfall & OData-Funktion  \\
	\midrule 	
	Allgemeine Testfälle: & \\
	\midrule
	TF1 & JSON-Format \\
	 	    	TF2 & Servicedokument \\
			TF3 & \$metadata\\
	\midrule
	Testfälle je Entität: & \\
	\midrule		
			TF4 & \$expand\\
			TF5 & Navigation\\
			TF6 & Navigation mit \$links\\
			TF7 & \$count \\
			TF8 & \$ordberby\\
			TF9 & \$skip\\
			TF10 & \$top\\
			TF11 & GetEntititySet\\
			TF12 & GetEntitity\\
			TF13 & \$filter\\
	\bottomrule 
\end{tabular} 
\caption{Testfälle - OData-Service}
\label{tab:testfälle-OData}
\end{table} 

\chapter{Implementierung}
\label{cha:Implementierung}
\autsection{Backend}{Azimi}
\subsection{Werkzeuge}

Für die verschiedenen Abschnitte im Entwicklungsprozess eines Gateway-Services stellt SAP folgende Werkzeuge zur Verfügung.

\subsubsection{OData Modeler}
\label{sec:OData-Modeler}
Das grafische Werkzeug zur OData-Service-Modellierung ist als Eclipse-Plugin in den \textit{SAP Mobile Platform Tools} enthalten. Entitäten werden inklusive (Navigations-)Eigenschaften und Abhängigkeiten definiert, siehe \autoref{fig:OData-Modeler}.

Die Service-Metadaten werden als XML-Datei exportiert und vom UI5-Mock-Server genauso importiert wie vom Service Builder des SAP Gateway. Eine direkte Verbindung zum SAP-System und Import/Export bestehender Services ist ebenso möglich. Ab der Service-Definition laufen Front- und Backend-Entwicklung unabhängig voneinander.

\begin{figure}[h]
	\centering
	\includegraphics[width=1.25\textwidth]{odata-modeler} 
	\caption[OData Modeler]{OData Modeler.}
	\label{fig:OData-Modeler}
\end{figure}

\subsubsection{SAP Gateway Service Builder}
Das zentrale Werkzeug zur Erstellung von OData-Services ist der SAP Gateway Service Builder (Transaktion \emph{SEGW}). Der Service Builder enthält alle Funktionalitäten, die für die Serviceentwicklung benötigt werden \cite[S.\ 187-189]{BoennenDreesFischerHeinzStrothmann2014}. 


\subsubsection{Gateway-Client}
\label{sec:gateway-client}
Zum Testen des OData-Services wird der sogenannte Gateway-Client (Transaktion \emph{/IWFND/GW\_CLIENT}) verwendet. Mit ihm lassen sich alle Funktionalitäten des zuvor erstellten OData-Services testen. Testfälle lassen sich definieren und in Testgruppen zusammenzufassen, um sie später erneut auszuführen \cite[S.\ 190-192]{BoennenDreesFischerHeinzStrothmann2014}.


\subsubsection{Fehlerprotokoll}
Zur Fehlerbehandlung gibt es das Fehlerprotokoll unter der Transaktion \emph{/IWFND/ERROR\_LOG}. Hier laufen alle Fehler auf, die beim Aufruf von Gateway-Services auftreten. Bei Bedarf ist es dank Integration des Gateway Clients möglich, eine fehlerhafte Abfrage erneut auszuführen \cite[S.\ 192-193]{BoennenDreesFischerHeinzStrothmann2014}.
%193 - Syslog und Anwendungsprotokoll?

\subsubsection{Services aktivieren und verwalten}
In der Transaktion \emph{/IWFND/MAINT\_SERVICE} werden neue Services registriert und aktiviert. Ausgewählte Services lassen sich im Browser oder Gateway-Client testen und das dazugehörige Fehlerprotokoll wird, wenn nötig, geöffnet \cite[S.\ 571-573]{BoennenDreesFischerHeinzStrothmann2014}.


\subsection{Funktionsbausteine}
Für den Zugriff auf Daten aus dem SAP-Backend werden in der zentralen Aufgabenverwaltung derzeit Funktionsbausteine verwendet. Diese werden remotefähig gemacht, um über die \ac{RFC}-Schnittstelle aufrufbar zu sein. Für die Abfrage der Daten werden die \ac{RFC}-Funktionsbausteine vom OData-Service mit entsprechenden Parametern aufgerufen. Die RFC-Funktionsbausteine kommen dann beim Lesen der Kunden-, Projekt-, Sprint-  und Aufgabendaten sowie der möglichen Aufgabenstatus der einzelnen Projekte zum Einsatz. 

\begin{listing}[H]
	\inputminted{abap}{src/fubaprojekte.abap}
	\caption{Z\_SCRUMUI5\_READ\_PROJEKTE}
	\label{lst:Quellcode-ReadProjekte}
\end{listing}

\subsubsection{Entwicklung}
In allen Funktionsbausteinen werden Importparameter definiert, die vom OData-Service übergeben werden können. Dieses sind in der Regel die Primärschlüssel der Datenbanktabellen, auf die durch einen Funktionsbaustein zugegriffen wird, um eindeutige Datensätze auslesen zu können. Zusätzlich wird ein Parameter zur Filterung nach Teil-Strings definiert. Die ausgelesenen Daten werden über eine Tabelle, mit der Struktur der ursprünglichen Datenbanktabelle, zurückgegeben.

Im Listing \ref{lst:Quellcode-ReadProjekte} wird der Funktionsbaustein zum Auslesen der Projektdaten dargestellt. Es gibt drei mögliche Szenarien zum Aufruf des Bausteins mit den jeweils nötigen Übergabeparametern:

\begin{enumerate}
	\item Alle Projekte eines Kunden (KundenID)
	\item Ein bestimmtes Projekt des Kunden (KundenID, ProjektID)
	\item Projekte des Kunden die Teil-String enthalten (KundenID, ProjektID)
\end{enumerate}
In den ersten beiden Szenarien werden einfache Select-Statements auf die Datenbanktabellen ausgeführt um auf die Daten zuzugreifen. Da mit dem SQL Like Operator im Open SQL die Nichtbeachtung von Groß- und Kleinschreibung nicht möglich ist, wird im letzten Szenario, nach dem Select-Statement, durch alle zurückgegebenen Datensätze gelaufen und überprüft ob der übergebene Teil-String in der Bezeichnung des Projektes vorhanden ist. Hierbei ist es dann egal, ob die Groß- und Kleinschreibung des Datenbankeintrages identisch mit der des Teil-Strings ist.


\subsubsection{Testen}
Zum Testen von neuen Funktionsbausteinen gibt es in der ABAP Workbench die Funktion den jeweiligen Funktionsbaustein auszuführen und zu testen. Dazu wählt man in der Drucktastenleiste \textit{Testen/Ausführen (F8)}. Man gelangt auf das Bild \textit{Funktionsbaustein testen: Eingabebild}. Hier werden alle Import-Parameter des Funktionsbausteins angezeigt. Nach Eingabe der Parameter kann der Funktionsbaustein mit \textit{Ausführen (F8)} gestartet werden. Der Funktionsbaustein wird mit den eingetragenen Importparametern ausgeführt und vorhandene Exportparameter oder Tabellen werden nach Ausführung angezeigt. 


\begin{figure}[h]
	\centering
	\includegraphics[width=.95\textwidth]{Testdatenverzeichnis} 
	\caption[Funktionsbaustein -- Testdatenverzeichnis]{Funktionsbaustein -- Testdatenverzeichnis.}
	\label{fig:Testdatenverzeichnis}
\end{figure}

\begin{figure}[h]
	\centering
	\includegraphics[width=.95\textwidth]{Regressionstest} 
	\caption[Funktionsbaustein -- Regressionstest]{Funktionsbaustein -- Regressionstest.}
	\label{fig:Regressionstest}
\end{figure}

Es ist möglich Testfälle im Testdatenverzeichnis zu speichern. Das Testdatenverzeichnis erreicht man über den ersten Eingabebildschirm. Alle gespeicherten Testfälle werden mit Erstelldatum und Kurztext wie in \autoref{fig:Testdatenverzeichnis} angezeigt und lassen sich hierüber erneut aufrufen. Bei Ausführung eines  Regressionstest wird im Ergebnis angezeigt ob sich das Ergebnis seit dem Speichern des Testes verändert hat, siehe \autoref{fig:Regressionstest}. Falls eine Differenz zum gespeichertem Ergebnis vorhanden ist, lässt sich diese anzeigen.\\ 



\subsection{OData-Service}
Für die Erstellung eines OData-Services im SAP Gateway gibt es generell unterschiedliche Wege: Zum einen gibt es die Serviceentwicklung mit \ac{ABAP}, welches flexiblere und effizientere Services ermöglicht. Diese erfordert allerdings spezielles \ac{ABAP}-Know-how. 

Alternativen dazu sind die Servicegenerierung durch den \ac{RFC}/\ac{BOR}-Generator, die Redefinition oder die Model Composition von bereits vorhandenen Gateway-Services. Die Servicegenerierung führt zu schnelleren Ergebnissen, ist durch eingeschränkte Rechte allerdings oft nicht weiter anpassbar.

Im Normalfall rechtfertigen die Vorteile durch die klassische ABAP-Service-Entwicklung den höheren Aufwand. Die Möglichkeit der Servicegenerierung wird interessant, sobald es geeignete Datenquellen für die automatische Generierung gibt \cite[S.\ 181]{BoennenDreesFischerHeinzStrothmann2014}.

Generell wird ein inkrementeller Serviceerstellungsprozess empfohlen, in dem der Service \bzw Teile davon direkt ausgeführt und getestet werden. Danach kann der Service so lange verändert werden, bis schließlich alle Anforderungen erfüllt werden \cite[S.\ 184]{BoennenDreesFischerHeinzStrothmann2014}.


\subsubsection{Entwicklung}
Die Entwicklung eines Gateway-Services lässt sich in drei Phasen einteilen:

\begin{quote}
	\begin{enumerate}
		\item Definition des Datenmodells
		\item Serviceimplementierung
		\item Serviceverwaltung
	\end{enumerate}
\end{quote}
%Definition des Datenmodells mit dem OData Modeler auf Basis der vorhandenen Struktur?
Zur \textit{Definition des Datenmodells} im SAP Gateway Service Builder gibt es unterschiedliche Varianten. 
Dazu zählen unter anderem die manuelle Deklaration des Datenmodells im Service Builder. Hierbei werden die Entitätstypen, Assoziationen und Assoziationsmengen manuell angelegt.

Alternativ kann ein bereits vorhandenes Datenmodell importiert werden, welches \zB mit dem OData Modeler erstellt wurde. Falls eine bereits vorhandene Datenstruktur aus einem SAP-System genutzt werden soll, können die Entitätstypen über einen DDIC-Import oder über den Import von \ac{RFC}/\ac{BOR}-Schnittstellen generiert werden \cite[S.\ 194-195]{BoennenDreesFischerHeinzStrothmann2014}.

\begin{figure}[h]
	\centering
	\includegraphics[width=.95\textwidth]{ImportRFC} 
	\caption[Import-RFC-Funktionsbaustein]{Import-RFC-Funktionsbaustein \cite[S.\ 45]{BoennenDreesFischerHeinzStrothmann2014}.}
	\label{fig:ImportRFC}
\end{figure}

In unserem Fall kommt der Import über RFC-Funktionsbausteine wie in \autoref{fig:ImportRFC} zum Einsatz, da diese auch für den Zugriff auf Daten aus dem SAP-Backend genutzt werden. Während des Imports des RFC-Funktionsbausteins müssen die Primärschlüssel der Entitätstypen festgelegt werden.

Unter den Eigenschaften der Entitätstypen lassen sich die einzelnen Attribute der Entitätstypen anzeigen und deren Eigenschaften ändern. So wird etwa definiert, ob ein bestimmtes Attribut sortier- oder filterbar ist. Diese Eigenschaften werden im Service-Metadatendokument hinterlegt. Diese haben allerdings nur einen informativen Charakter: Es erfolgt keine weitere Überprüfung durch das Gateway-Framework. So können Attribute sortiert werden, für die keine Sortier-Eigenschaft gesetzt wurde. 


Als nächstes müssen die Entitätsmengen definiert werden, die im Regelfall in einer 1:1-Beziehung zu den Entitätstypen stehen. Der spätere Zugriff über den OData-Service läuft immer über die Entitätsmengen und nicht über die Entitätstypen. Auch hier gibt es einige Attribute zu pflegen, \zB, ob ein Eintrag eines Entitätstypen angelegt, aktualisiert oder gelöscht werden darf, oder ein Filter für die Abfrage der Entitätsmenge zwingend erforderlich ist. Wie auch die Annotationen der Entitätstypen sind auch diese nicht zwingend bindend und erforderlich, sollten jedoch passend zum tatsächlichen Funktionsumfang des OData-Services gepflegt werden. So ist es für Service-Konsumenten einfacher herauszufinden, welchen Funktionsumfang der OData-Service unterstützt \cite[S.\ 237-240]{BoennenDreesFischerHeinzStrothmann2014}.


Anschließend wird der Service im SAP-System registriert. Hier werden die Laufzeitobjekte, die für den Gateway-Service benötigt werden, generiert \cite[S.\ 198]{BoennenDreesFischerHeinzStrothmann2014}.
Danach geht es an die \textit{Serviceimplementierung}. Hier gibt es die Auswahl zwischen der Implementierung durch ABAP-Programmierung  oder das Mappen von \ac{RFC}/\ac{BOR}-Schnittstellen \cite[S.\ 201-202]{BoennenDreesFischerHeinzStrothmann2014}.

\piccaption[RFC-Mapping]{RFC-Mapping\label{fig:MappingRFC}.}
\parpic{
	\includegraphics[width=.5\textwidth]{MappingRFC}
}

Um die Attribute einer En\-ti\-täts\-men\-ge mit einem RFC-Funk\-tions\-bau\-stein zu mappen, wählt man \emph{Zu Da\-ten\-quel\-le zuordnen} für den gewünschten Vorgang einer En\-ti\-täts\-men\-ge, hier: GetEntitySet der En\-ti\-täts\-men\-ge Kunden, siehe \autoref{fig:MappingRFC}.

Nach Angabe des Funk\-tions\-bau\-steins können die Attribute der Quelle zugeordnet werden. 

Dies funktioniert manuell oder über den Button \emph{Zuordnung vorschlagen} wie in \autoref{fig:MappingRFC2}. Zu diesem Zeitpunkt ist lediglich das Abrufen von Daten ohne zusätzliche Funktionen möglich. Um die Filterfunktionen des OData-Services nutzen zu können, müssen die Import-Parameter der Funktionsbausteine zugeordnet werden.

Dafür wird eine neue Zeile mit der Entitätsmenge des Importparameters hinzugefügt -- In diesem Fall \textit{Kunde}. Dieser wird dem Import-Parameter aus der Datenquelle zugeordnet. Jetzt kann nach bestimmten Kunden gefiltert werden.
So lassen sich Services mit Grundfunktionen relativ einfach durch die Servicegenerierung erstellen. Sobald Funktionen wie das Filtern nach Teil-Strings benötigt werden, ist eine Zusatzimplementierung durch ABAP-Programmierung notwendig.



\begin{figure}[h]
	\centering
	\includegraphics[width=.95\textwidth]{MappingRFC2} 
	\caption[RFC-Mapping -- Zuordnung der Attribute]{RFC-Mapping -- Zuordnung der Attribute.}
	\label{fig:MappingRFC2}
\end{figure}

Falls der Service durch ABAP-Programmierung erstellt werden soll, müssen die Methoden für den Vorgang in der ABAP-Workbench redefiniert werden. Die bei der Servicegenerierung automatisch erstellte Methode wird hier selbst implementiert. Alle Parameter, die vom OData-Service an das Gateway gesendet werden, müssen berücksichtigt und die entsprechenden Funktionen implementiert werden.

In \autoref{tab:ODataQueryOptions} gibt es eine kurze Übersicht, der vom SAP Gateway unterstützen OData-Query-Optionen, wovon einige eine zusätzliche Implementierung benötigen.

\begin{table}[h]
	\centering
	\begin{tabular}{cc}
		\toprule OData-Query-Optionen & Zusätzliche Implementierung notwendig \\ 
		\midrule \$select & Nein\\
		\$count & Nein\\
		\$expand & Nein\\
		\$format & Nein\\
		Read \$links & Nein\\
		\$value & Nein\\
		\$orderby & Ja\\
		\$top & Ja\\
		\$skip & Ja\\
		\$filter & Ja\\
		\$inlinecount & Ja\\
		\$skiptoken & Ja\\
		\bottomrule 
	\end{tabular} 
	\caption{Unterstützte OData-Query-Optionen im SAP Gateway aus \cite{SAP2015_3}}
	\label{tab:ODataQueryOptions}
\end{table} 



\begin{listing}[H]
	\inputminted{abap}{src/kundengetentityset.abap}
	\caption{KUNDEN\_GET\_ENTITYSET}
	\label{lst:Quellcode}
\end{listing}

In Listing \ref{lst:Quellcode} wird, falls in der Abfrage angegeben, ein Teil-String der Kundenbezeichnung in einer Variablen gespeichert. Diese wird anschließend an den Funktionsbaustein übergeben. Je nach Abfrage werden alle Kunden oder nur Kunden mit Teil-String im Namen zurückgegeben. Zum Abschluss werden die zurück erhaltenen Daten in die Tabelle \emph{ET\_ENTITYSET} geschrieben und an den OData-Service übergeben.

Nach Implementierung des Services muss dieser noch über die \textit{Serviceverwaltung} beim Gateway-Server registriert und aktiviert werden. Anschließend kann der OData-Service getestet werden.
Sind voneinander abhängige Entitätstypen im Service vorhanden, werden die Relationen mit den entsprechenden Kardinalitäten zwischen den Entitätstypen in den Assoziationen wie in \autoref{fig:Assoziationen} definiert. Die zuvor erstellten Assoziationen werden anschließend den Entitätstypen als Navigationseigenschaften zugewiesen, siehe \autoref{fig:Navigationseigenschaften}.

\begin{figure}[h]
	\centering
	\includegraphics[width=.95\textwidth]{Assoziationen} 
	\caption[Entitätstypen -- Assoziationen]{Assoziationen zwischen den Entitätstypen.}
	\label{fig:Assoziationen}
\end{figure}


\begin{figure}[h]
	\centering
	\includegraphics[width=.75\textwidth]{Navigationseigenschaften} 
	\caption[Entitätstypen -- Navigationseigenschaften]{Navigationseigenschaften zwischen den Entitätstypen.}
	\label{fig:Navigationseigenschaften}
\end{figure}


\subsubsection{Testen}

\begin{figure}
	\centering
	\includegraphics[width=.95\textwidth]{Testfalldatenbankodata} 
	\caption[OData-Testfalldatenbank]{OData-Testfalldatenbank.}
	\label{fig:Testfalldatenbankodata}
\end{figure}


\begin{figure}
	\centering
	\includegraphics[width=.95\textwidth]{Mehrfachtest} 
	\caption[OData-Mehrfachtest]{OData-Mehrfachtest.}
	\label{fig:Mehrfachtest}
\end{figure}

Ist der OData-Service erstellt und registriert, geht es an das Testen mit dem SAP Gateway Client. Wie bereits in \autoref{sec:gateway-client} erwähnt, lassen sich die Testfälle für den OData-Service in Testgruppen speichern, und jederzeit wieder abspielen, siehe \autoref{fig:Testfalldatenbankodata}. Das Abspielen geschieht in einer sequenziellen Reihenfolge, um so komplexe Szenarien mit Erstellung, Update, Lesen und Löschen von Daten zu testen. Zu jedem abgesicherten Testfall lässt sich ein erwarteter HTTP-Statuscode definieren, wie in etwa der Code \textit{200} für den Status \textit{OK} oder \textit{4xx} für Client-Fehler \cite[S.\ 555-556]{BoennenDreesFischerHeinzStrothmann2014}.


Nach Ausführung der Testfälle gibt es eine Übersicht wie in \autoref{fig:Mehrfachtest}, inklusive erwartetem und tatsächlichem HTTP-Statuscode, sowie möglichem Fehlertext. Durch die Integration des Fehlerprotokoll lässt sich über die Auswahl eines fehlgeschlagenen Tests direkt in das Fehlerprotokoll wechseln, um die Fehlerursache zu ermitteln. 

Zusätzlich kann im Browser die Query-Option \textit{sap-ds-debug=true} genutzt werden, um Antworten des Services im Browser als HTML-Seite anzeigen zu lassen. Neben der Rückantwort gibt es zusätzliche Informationen wie \zB die Laufzeitanalyse der ABAP Methoden. Falls Fehler im Service auftreten, erscheint der Tab \textit{Stacktrace}, in dem die auftretende Exception angezeigt wird \cite[S.\ 555-556]{BoennenDreesFischerHeinzStrothmann2014}.


%\section{Ergebnisse}

\section{CI-Toolchain}
%Kurze Vorstellung aller Tools? 
%NodeJS
%Bower
%PhonegapBuild oder ans Ende hinter Jenkins?
%Karma
%Grunt
%Jenkins
%Zusammenfassung?
\autsubsection{Paketmanager}{Mattfeld}

Die Paketmanager sind existenzieller Bestandteil der \ac{CI}-Toolchain -- Das Prinzip \textit{Infrastructure as Code} ist nur mit ihrer Hilfe umzusetzen: Die Abhängigkeiten müssen nicht mehr in die Versionskontrolle eingecheckt werden, stattdessen werden die benötigten Versionen bei Bedarf extern nachgeladen.

\subsubsection{Node.js}
Node.js ist eine JavaScript-Laufzeitumgebung und für diverse Plattformen verfügbar. Basis ist die schnelle JavaScript-Engine Google V8, die um Techniken wie Nebenläufigkeit und nicht-blockierende I/O-Zugriffe erweitert wurde. Node.js kann eine große Zahl an gleichzeitigen Netzwerkverbindungen verwalten, ein zusätzlicher HTTP-Server ist nicht erforderlich. Ursprünglich sollte Node.js den einfachen Betrieb von Echtzeit-Webanwendungen ermöglichen.

Für dieses Projekt nutzen wir Node.js nicht als Server, sondern machen uns das stark gewachsene Ökosystem in Form des Node.js Package Managers \textit{npm} zu nutze. npm ist dabei zum einen ein lokales Tool zur Paketverwaltung, das direkt mit Node.js ausgeliefert wird, zum anderen gibt es ein zentrales Repository mit downloadbaren Paketen.

Diese Prinzip fördert eine hohe Modularisierung: JavaScript-typisch existieren viele kleine Tools, die eine bestimmte Aufgabe lösen, siehe \autoref{sec:tddjs}. Alle im Projekt genutzten Entwicklungs- und Testtools beziehen wir über npm.
Die Installation erfolgt nicht global, sondern lokal im Projektverzeichnis \textit{node\_modules}. Für verschiedene Projekte kann das gleiche Tool ohne Probleme in verschiedenen Versionen installiert werden. 

Formal ist auch unser UI5-Projekt selbst ein Node.js-Paket. Die Metadaten in der Datei \textit{package.json}, siehe Listing \ref{lst:package.json}, beschreiben es mit Namen und Version. Wir definieren zusätzlich alle Entwicklungsabhängigkeiten. Über den Befehl \textit{npm install} werden diese automatisch in der angegebenen Version installiert. Auch weitere, indirekte Abhängigkeiten werden aufgelöst. Ein neues Tool wird über \textit{npm install package -{}-save-dev} installiert und automatisch, als weitere Abhängigkeit für zukünftige Installationen, gespeichert.

Der Zusatz \textit{private} in der Konfiguration stellt sicher, dass wir das Paket nicht unabsichtlich über \textit{npm} veröffentlichen. Zusätzlich können private Repositories erstellt werden. So ist \zB eine firmeninterne, projektübergreifende Nutzung von Komponenten möglich.

Als Best Practice haben sich feste Versionen der Entwicklungsabhängigkeiten erwiesen. Im Gegensatz zu einem echten Node.js-Projekt sind sie bei uns nur ein Werkzeug und müssen keine breite Kompatibilität mit externen Abhängigkeiten eines veröffentlichten Pakets gewährleisten. Eine einmal funktionierende Konfiguration sollte beibehalten werden, um den Testaufwand der Infrastruktur gering zu halten.

\begin{listing}[H]
	\inputminted{json}{src/package.json}
	\caption{package.json (gekürzt)}
	\label{lst:package.json}
\end{listing}

\subsubsection{Bower}
Der zweite Paketmanager Bower ist nur für Frontend-Frameworks zuständig. In diesem Fall installiert er die UI5-Kernkomponenten \textit{sap.ui.core, sap.m} und das Standardtheme \textit{sap\_bluecrystal} im Projektordner unter \textit{bower\_\-components}. Sie sind zwar als Entwicklungsabhängigkeiten aufgeführt, wir nutzen sie aber gleichzeitig für das Release der App. Ein Grunt-Task kopiert die minified-Versionen in den Target-Ordner. Der Verwaltungs- und Testaufwand wird minimiert, weil für Entwicklung und Release die gleiche Version genutzt wird. Im Gegensatz zum offiziellen OpenUI5-Paket sind beim Bezug über Bower außerdem die Test-Frameworks QUnit und OPA5 enthalten.

Die automatische Paketinstallation wird über \textit{bower install} angestoßen. Auch die Paket-Verwaltung erfolgt synchron zu npm zentral über die Datei \textit{bower.json}, siehe Listing \ref{lst:bower.json}.
Im Unterschied zu npm unterstützt Bower keine geschachtelten Abhängigkeiten. Jedes Framework existiert nur einmal im Projekt -- Der Verzeichnisbaum hat nur eine Ebene. Bower arbeitet dadurch schneller als npm. 
Die einfache Struktur zwingt außerdem dazu, eine kompatible Framework-Kombination zu finden. Für den Endbenutzer führt dies zu schnelleren Ladezeiten, da nur jeweils eine Version heruntergeladen werden muss.

Mittelfristig wird Bower durch den mächtigeren npm abgelöst. Der Pflegeaufwand für zwei Paketmanager ist hoch. Viele Projekte haben dies erkannt, und bieten auch ihre Frontend-Frameworks per npm an. Bei OpenUI5 ist dies aber noch nicht der Fall.

\begin{listing}[H]
	\inputminted{json}{src/bower.json}
	\caption{bower.json (gekürzt)}
	\label{lst:bower.json}
\end{listing}

\autsubsection{Statische Analyse mit ESLint}{Mattfeld}
Da JavaScript nicht kompiliert wird, werden Syntax-Fehler erst während der Ausführung erkannt. Die statische Code-Analyse setzt schon vorher, während der Bearbeitung im Editor, an. Häufige Fehler oder sogenannte Anti-Patterns (nicht empfohlene Vorgehensweisen) sind global deklarierte oder nie genutzte Variablen. Außerdem lassen sich bestimmte Code-Styleguides durchsetzen.

ESLint ist ein Open-Source-Projekt auf Node.js-Basis. Die Installation erfolgt entsprechend per npm. ESLint bringt bereits ein vordefiniertes Regelset mit. Im Gegensatz zu JSLint oder JSHint ist dieses jedoch vollständig frei konfigurierbar. In der Datei \textit{.eslintrc} werden die vorhandenen Regeln als Warnung oder Fehler eingestuft -- oder ignoriert. Globale Variablen sind frei definiert oder umgebungsspezifisch als gültig deklariert, \zB im Node.js- oder jQuery-Kontext.

Wir übernehmen das Regelset aus OpenUI5. Zusätzlich nutzen wir das Plugin \textit{eslint-openui5}, das eine weitere UI5-spezifische Regel für Zeilenenden hinzufügt. Besonderheit: Weder das UI5-Framework, noch die Best-Practice-Beispiele sind mit den Regeln konform. \ZB das mittlerweile obligatorische \textit{'use strict';} fehlt. Das von SAP veröffentlichte Regelset ist eher Ziel, als aktueller Standard. Entsprechend unterziehen wir das Framework keiner statischen Prüfung. Unsere Eigenentwicklungen sind allerdings ESLint-konform, um zukünftige Kompatibilität zu gewährleisten.

\autsubsection{Testtreiber Karma}{Mattfeld}
\label{sec:karma}
Karma ist ein JavaScript-Testtreiber auf Node.js-Basis und aus dem Google-Projekt AngularJS hervorgegangen. Er ist das Verbindungsglied zwischen Test-Frameworks (Qunit, OPA5), CI-Tool (Grunt/Jenkins) und Browser (Desktop, Mobil, Headless, \dots). Die Architektur ist modular: Verbindungsfunktionen übernehmen Plugins, sogenannte Karma-Adapter.

Installiert wird karma per npm. Anschließend kann die Erstkonfiguration über \textit{karma init karma.conf.js} gestartet werden. Ein Assistent führt zu einer Minimalkonfiguration wie in Listing \ref{lst:karma.conf.js}. Wir benötigen die Test-Frameworks QUnit und OPA5. Außerdem speichern wir den Pfad zum OpenUI5-Framework und aktivieren dessen Mockserver. 

Beim Start per \textit{karma start}, bzw. über die IDE oder Jenkins/Grunt, stellt Karma bestimmte Dateien über seinen integrierten Server bereit. Die verschiedenen File-Anweisungen in der Konfiguration schließen OpenUI5, unsere App und die Tests ein (\textit{served}). Der Parameter \textit{watched} überwacht das angegebene Verzeichnis. Ändern sich App-Code oder Testdaten, wird ein erneuter Testlauf angestoßen. \textit{included} markiert die Verzeichnisse mit Testskripts.

Die weiteren Parameter aktivieren eine Code-Coverage-Protokollierung, automatische Regressionstests und den zu öffnenden Browser. Möglich wären hier statt Chrome auch diverse andere Desktop-Browser, mobile Geräte und ein Test ohne grafische Oberfläche per PhantomJS. \textit{singleRun} ist eine spezielle CI-Option: Karma beendet Server und Browser automatisch nach einem Testlauf.

\begin{listing}[H]
	\inputminted[tabsize=2]{js}{src/karma.conf.js}
	\caption{karma.conf.js -- Minimalkonfiguration}
	\label{lst:karma.conf.js}
\end{listing}

Speziell für unser UI5-Projekt benötigen wir den Adapter \textit{karma-openui5}\footnote{\url{https://github.com/SAP/karma-openui5}}. Er lädt OpenUI5 aus dem bereits angegebenen Verzeichnis. Zusätzlich werden wie in Listing \ref{lst:karma.openui5.conf.js} die weiteren Bibliotheken und Themes bereitgestellt. Die Konfiguration erfolgt analog zum UI5-Bootstrapping.

\begin{listing}[H]
	\inputminted[tabsize=2]{js}{src/karma.openui5.conf.js}
	\caption{karma.conf.js -- Ausschnitt OpenUI5}
	\label{lst:karma.openui5.conf.js}
\end{listing}

Der Mockserver wird wie in Listing \ref{lst:karma.mockserver.conf.js} konfiguriert. \textit{rootURI} entspricht der OData-Service-Adresse, die unsere App aufruft. Der Mockserver fängt entsprechende Aufrufe ab und stellt einen, anhand der \textit{metadata.xml} definierten, Service bereit. Diese kann per OData Modeler erstellt, oder von einem bereits verfügbaren Service exportiert werden.

Unter \textit{mockdataSettings} ist das Verzeichnis der Mock-Daten angegeben, also die Objekte, die der Server zurückliefert. In diesem Projekt sind das \zB \textit{Kunden.json} oder \textit{Projekt.json}, die wir aus dem laufenden Service exportiert haben. Alternativ erzeugt der Mockserver auch Zufallsdaten aus der Spezifikation.

\begin{listing}[H]
	\inputminted[tabsize=2]{js}{src/karma.mockserver.conf.js}
	\caption{karma.conf.js -- Ausschnitt Mockserver}
	\label{lst:karma.mockserver.conf.js}
\end{listing}

Zur Zeit (Stand: 12.03.2015) ist der Adapter karma-openui5 nicht ohne Modifikation in unserer Toolchain lauffähig. Um die Mockserver-Funktion zu nutzen, muss die Datei \textit{./node\_modules/karma-openui5/lib/mockserver.js} in Zeile 7 erweitert werden. Listing \ref{lst:mockserver.js} zeigt den aktuellen Workaround. Das Problem ist gemeldet und in Klärung.

\begin{listing}[H]
	\inputminted[tabsize=2]{js}{src/mockserver.js}
	\caption{mockserver.js Erweiterung}
	\label{lst:mockserver.js}
\end{listing}

\autsubsection{Hybrid-App per PhoneGap Build}{Azimi}
PhoneGap Build\footnote{\url{http://build.phonegap.com}} ist ein Cloud-Service von Adobe um mobile Web-An\-wen\-dun\-gen auf Basis des Open Source Toolkits Apache Cordova in der Cloud zu kompilieren. 

\begin{figure}[h]
\centering
\includegraphics[width=.95\textwidth]{PhoneGapBuild}
\caption[PhoneGap-Build-Webinterface]{PhoneGap-Build-Webinterface.}
\label{fig:Phonegap}
\end{figure}

Die Konfiguration des PhoneGap Builds läuft über die Weboberfläche, die auf \autoref{fig:Phonegap} zu sehen ist. Der Quellcode der Web-Applikation kann als ZIP-Archiv gepackt und hochgeladen oder automatisch aus einem Git-Repository geladen werden. Anschließend wird die Anwendung für verschiedene Betriebssysteme direkt kompiliert und signiert. Für die Signierung der iOS-App müssen die  entsprechenden Entwicklerzertifikate hinterlegt werden.

Nach erfolgreichem Build der Anwendung können die Installationsdateien für die verschiedenen Plattformen heruntergeladen und für die weitere Verteilung genutzt werden. Zudem ist es möglich, die Anwendung über einen Link zu verteilen. Dieser installiert auf mobilen Geräten automatisch die App. 

Auf native Gerätefunktionen wird durch verschiedene Plugins zugegriffen, sodass Kamera oder GPS-Koordinaten auch für Web-Apps zugänglich sind. Durch das Kompilieren in der Cloud wird kein spezielles \ac{SDK} für die verschiedenen Plattformen wie iOS und Android benötigt. Es reicht eine Entwicklungsumgebung für Webanwendungen.

\autsubsection{JS Task Runner Grunt}{Azimi}
Um die Vielzahl an Werkzeugen für unseren automatischen Build aufzurufen, kommt der sogenannte JavaScript Task Runner Grunt\footnote{\url{http://gruntjs.com/}} zum Einsatz. Einzige Aufgabe ist die Abarbeitung der einzelnen Tasks, die in der Grunt-Konfiguration (siehe Listing \ref{lst:gruntfile}) definiert werden \cite[S.\ 59-61]{Wrobel2015}. Die eigentlichen Funktionalitäten sind in diverse Grunt-Plugins ausgelagert, die wir im Folgenden erläutern.

\begin{listing}
	\inputminted{javascript}{src/gruntfile-short.js}
	\caption{Auszug Gruntfile.js}
	\label{lst:gruntfile}
\end{listing}

\subsubsection{grunt-contrib-uglify}
Dies ist ein Grunt-Plugin zum Ausführen des UglifyJS-Toolkits. Hiermit wird der JavaScript-Code komprimiert, Variablennamen werden, wenn möglich, verkürzt und vorhandene Kommentare gelöscht. Die minified-JavaScript-Files besitzen typischerweise nur noch 30--40\,\% der Ursprungsgröße.
%  \cite{Uglify2015}

\subsubsection{grunt-eslint}
Überprüft den JavaScript-Code auf Einhaltung der vereinbarten Regeln zur statischen Code-Analyse per ESLint, einer Weiterentwicklung von JSHint. Hierfür stellt SAP im OpenUI5-Repository eine ESLint-Konfigurationsdatei zur Verfügung, die in eigenen Projekten verwendet werden kann \cite{SAP2015_2}.

\subsubsection{grunt-karma}
Dieses Plugin ermöglicht es, Karma während des Build-Prozesses aufzurufen. Über separate Karma-Tasks werden die Unit- und Akzeptanztests ausgeführt. \cite[S.\ 129-130]{Wrobel2015}.

\subsubsection{grunt-contrib-copy}
Verschiebt Dateien, \zB vom Source- in den Destination-Ordner.

\subsubsection{grunt-contrib-compress}
Komprimiert den Target-Ordner in einer ZIP-Datei, damit diese anschließend vom PhoneGap-Build-Plugin weiterverarbeitet wird.

\subsubsection{grunt-phonegap-build}
Lädt das in einer ZIP-Datei archivierte Projekt über ein \ac{API} zu PhoneGap Build. Anschließend löst es einen neuen Build aus und signiert die Installationsdateien.

\subsubsection{grunt-plato}
Neben der statischen Code-Analyse durch ESLint gibt es weitere Analysen des Quelltextes durch das plato-Plugin. Es stellt verschiedene Metriken zur Komplexität und Wartungsfreundlichkeit des Codes bereit.
%https://github.com/jsoverson/grunt-plato
%http://de.slideshare.net/JarrodOverson/complexity-28214103
%HIER MEHR SCHREIBEN!
\SuperPar
\SuperPar
\autsubsection{Jenkins}{Azimi}
Als Continuous-Integration-Server wird Jenkins\footnote{\url{https://jenkins-ci.org/}} eingesetzt. Für die Implementierung verschiedener Werkzeuge im Continuous-Integration-Prozess werden zusätzliche Plugins wie folgt verwendet:

\begin{figure}[h]
\centering
\includegraphics[width=.50\textwidth]{Github2-JenkinsConfig}
\caption[Build-Auslöser]{Build-Auslöser.}
\label{fig:Github2}
\end{figure}

\subsubsection{GitHub-Plugin}Zur Versionsverwaltung der Projektdaten wird GitHub verwendet. Das Plugin stößt den Build-Prozess bei Code-Änderungen im Repository automatisch an. Ebenfalls möglich ist eine zeitgesteuerte Ausführung oder ein manueller Start wie in \autoref{fig:Github2}. So ist gewährleistet, dass der Build inkl. Tests mit der aktuellen Version des Codes durchgeführt wird.

\subsubsection{Einbindung Paketmanager und Grunt}
Die Einbindung von Node.js, Bower und Grunt in den Continious-Integration-Prozess funktioniert durch Kommandozeilenbefehle, siehe \autoref{fig:Kommandozeile}, welche von Jenkins ausgeführt werden.

\begin{figure}[h]
\centering
\includegraphics[width=.95\textwidth]{Kommandozeile}
\caption[Integration von Paketmanagern und Grunt]{Integration von Paketmanagern und Grunt.}
\label{fig:Kommandozeile}
\end{figure}


\subsubsection{Checkstyle}
Um die Ergebnisse des ESLint-Checks in Jenkins anzeigen zu können, wird das Jenkins-Checkstyle-Plugin genutzt. Dazu muss in der Jenkins-Projekt-Konfiguration der Pfad zu den Ergebnissen angegeben werden. Anschließend wird in der Projektübersicht ein Trend-Graph, wie in \autoref{fig:Checkstyle-Trendgraph} zu sehen ist, mit der Anzahl an Checkstyle-Warnungen erstellt. Zusätzlich lassen sich diese Warnungen je nach Verzeichnis, Datei oder Warnungs-Typ aufschlüsseln, vergleiche \autoref{fig:Checkstyle-Warnungen}.

\begin{figure}[h]
\centering
\includegraphics[width=.95\textwidth]{Checkstyle-Trendgraph} 
\caption[Checkstyle-Warnungen -- Trend-Graph]{Trend-Graph der Checkstyle-Warnungen.}
\label{fig:Checkstyle-Trendgraph}
\end{figure}

\begin{figure}[h]
\centering
\includegraphics[width=.95\textwidth]{Checkstyle-Warnungen-high}
\caption[Checkstyle-Warnungen -- Hohe Priorität]{Checkstyle-Warnungen -- Hohe Priorität.}
\label{fig:Checkstyle-Warnungen}
\end{figure}

\subsubsection{Cobertura}
Zur Anzeige der Zeilenabdeckung aus den Akzeptanztests wird das Cobertura Plugin gebraucht. Das Plugin erzeugt einen Trend-Graphen, in dem die Zeilenabdeckung des Codes untergliedert in Pakete, Dateien, Klassen, Methoden, Zeilen und Verzweigung dargestellt wird, siehe \autoref{fig:CodeCoverage-Package}. Über den Graphen kann bis in die einzelnen Dateien und den dazugehörigen Quellcode navigiert werden. Welche Teile des Codes durch die ausgeführten Tests bisher abgedeckt bzw. eben nicht abgedeckt sind, wird hier nachvollziehbar, wie in \autoref{fig:CodeCoverage-Code} zu sehen ist.

\begin{figure}[h]
\centering
\includegraphics[width=1.25\textwidth]{CodeCoverage-Package}
\caption[Zeilenabdeckung im Paket \emph{app}]{Zeilenabdeckung im Paket \emph{app}.}
\label{fig:CodeCoverage-Package}
\end{figure}

\begin{figure}[h]
\centering
\includegraphics[width=1.25\textwidth]{CodeCoverage-Quellcode}
\caption[Zeilenabdeckung im Quellcode]{Zeilenabdeckung im Quellcode.}
\label{fig:CodeCoverage-Code}
\end{figure}

\subsubsection{HTML Publisher}
grunt-plato stellt die Ergebnisse der statischen Code-Analyse als HTML-Bericht zur Verfügung. Durch das Plugin wird in der Jenkins-Projektübersicht auf diesen Plato-Bericht verlinkt. Der Bericht zeigt Ergebnisse für das komplette Projekt oder auch für einzelne Dateien. In \autoref{fig:Platoresult} sieht man einen Report für die Datei Master3.controller.js.

\begin{figure}[h]
\centering
\includegraphics[width=.95\textwidth]{Plato2-JenkinsReport}
\caption[Plato-Bericht zu \emph{Master3.controller.js}]{Plato-Bericht zu \emph{Master3.controller.js}.}
\label{fig:Platoresult}
\end{figure}

\subsubsection{Timestamper}
Fügt der Konsolenausgabe von Jenkins einen Zeitstempel hinzu.

\subsubsection{Workspace Cleanup}
Ermöglicht das automatische Löschen des Arbeitsbereiches vor Ausführung des Builds. Da der komplette Prozess als Code abgebildet ist, wird permanent eine neue Build-Umgebung geschaffen.

% \begin{wrapfigure}[lines]{position}{breite}
\begin{wrapfigure}[4]{R}{2.5cm}
	\centering
	\includegraphics{build_passing} 
	\caption[Build-Status]{Build-Status.}
	\label{fig:BuildStatus}
\end{wrapfigure}

\subsubsection{embeddable-build-status}
Erstellt ein Icon wie in \autoref{fig:BuildStatus} mit dem aktuellen Status des Builds zur Einbindung in Webseiten -- In unserem Fall in die GitHub Readme.

% Bugfix, Zeilennummern der Code-Listings sonst verschoben!
\begin{wrapfigure}[0]{l}{0cm}
\end{wrapfigure}

\newpage
\autsubsection{Zusammenfassung}{Azimi}
Die Testumgebung setzt sich aus drei großen Teilschritten zusammen. Diese sind die Build-Auslöser, der eigentliche Build-Prozess sowie die Post-Build-Operationen.

Build-Auslöser sind Ereignisse, die Jenkins dazu veranlassen einen neuen Build zu erstellen und alle Tests durchzuführen, die für dieses Projekt definiert wurden.
Dabei unterscheiden wir unter folgenden Build-Auslösern:
\begin{enumerate}
\item Der regelmäßige Build - jeden Tag um Mitternacht
\item Der Build bei Änderungen im GitHub - bei jedem Push
\item Der manuelle Start eines Buildvorganges
\end{enumerate}

Nach dem gemeldeten Ereignis durch den Auslöser startet Jenkins den Build. Zu Beginn jedes Builds wird der Arbeitsbereich gelöscht, um eine neue Build-Umgebung zu schaffen und die aktuelle Version des Quellcodes aus dem GitHub Repository herunterzuladen. Anschließend werden die benötigten Werkzeuge und Bibliotheken heruntergeladen und Grunt gestartet.

\SuperPar Im Build-Prozess werden schrittweise folgende Schritte ausgeführt:

\begin{enumerate}
\item Löschen des Arbeitsbereiches für einen sauberen Build
\item Code-Analyse durch ESLint und Plato
\item QUnit-Testausführung mit Karma inklusive Überprüfung der Zeilenabdeckung durch die QUnit bzw. OPA5-Tests
\item Kopieren der, für die App, notwendigen Dateien und Programmbibliotheken in den Target-Ordner
\item Komprimierung des JavaScript-Codes durch Uglify
\item Archivierung des Target-Ordners in eine ZIP-Datei
\item Hochladen der ZIP-Datei zu PhoneGap Build zum Kompilieren der Anwendung
\end{enumerate}

Zum Abschluss werden aus den Ergebnissen der verschiedenen Tests Berichte erzeugt und über die Jenkins-Oberfläche bereitgestellt. Die fertig kompilierte Anwendung lässt sich nun über PhoneGap Build auf den Geräten installieren. Die Dateistruktur des Projektes sieht nun so aus:

\begin{itemize}
	\item app/
	\item bower\_components/
	\item node\_modules/
	\item target/	
	\item test/
	\item test-reports/
	\item .eslintrc
	\item .gitignore
	\item bower.json
	\item config.xml
	\item Gruntfile.js
	\item karma.conf.js	
	\item package.json
\end{itemize}
\autsection{UI5-App}{Mattfeld}
\subsection{Auswahl der Entwicklungsumgebung}
\label{sec:auswahl_ide}

\subsubsection{Das bietet SAP}
Die von der SAP vorgeschlagene Universal-Entwicklungsumgebung ist Eclipse. Sie gehört \textit{noch} zur Firmenstrategie und wird mit einigen Plugins\footnote{\url{https://tools.hana.ondemand.com/}} einsatzbereit für die Entwicklung im SAP-Umfeld gemacht. Sie unterstützt unter anderem 
\begin{itemize}
	\item ABAP Development Tools for SAP NetWeaver
	\item SAP HANA Cloud Platform Tools
	\item Gateway (SAP Mobile Platform Tools)
	\item UI Development Toolkit for HTML5
\end{itemize}
Über die Transaktion \textit{/UI5/UI5\_REPOSITORY\_LOAD} werden UI5-Apps als \ac{BSP} auf einem SAP Web Application Server bereitgestellt. Das ist die Standard-Methode für Fiori-Apps. Die Transaktion ist remotefähig und wird per HTTP-Aufruf in eine CI-Toolchain eingebunden. Die gleiche Deploymentfunktion in Eclipse stellt der ABAP Repository Team Provider bereit.

Nicht zwingend notwendig, aber nützlich für die Backend-Entwicklung, ist der grafische OData Modeler aus den SAP Mobile Platform Tools, siehe \autoref{sec:OData-Modeler}. Er erleichtert Kommunikation und Abstimmung zwischen den Entwicklern. Die generierten Metadaten können sowohl im Gateway, als auch in SAPUI5 genutzt werden, ohne dass der eigentliche Service schon implementiert sein muss.

Das UI Development Toolkit for HTML5 enthält für die Frontend-Entwicklung:
\begin{itemize}
	\item App-Templates
	\item Test-Webserver
	\item Code-Vervollständigung
	\item XML-Syntaxprüfung
\end{itemize}
Ein schneller Einstieg in die UI5-Entwicklung ist mit diesen Tools möglich, sie sind jedoch keine Alleinstellungsmerkmale für Eclipse mit Plugin als \ac{IDE}. Sie sind auch in anderen Umgebungen verfügbar. Vielmehr sind die Templates für größere Projekte oder Nutzung als Fiori App durch fehlende Modularisierung sogar ungeeignet.

Noch schneller gelingt der Start mit der neuen SAP Web IDE\footnote{\url{http://scn.sap.com/docs/DOC-55465}}. Die Umgebung basiert auf Eclipse Orion und ist nach dem Login direkt startbereit. Dort sind modulare Templates für Fiori- und Hybrid-Apps verfügbar, die den aktuellen Best Practices folgen. Zu den weiteren Funktionen zählen:
\begin{itemize}
	\item Grafischer Layout-Editor 
	\item Live-Vorschau
	\item Code-Vervollständigung
	\item HANA Cloud Platform und ABAP-Repositories
	\item Mock-Daten-Support
\end{itemize}
Die ersten Erfahrungen mit der Web IDE sind vielversprechend: Die Anwendung reagiert schnell und implementiert außerdem anpassbare Code-Checks per ESLint und git-Support. Die Entwicklung schreitet schnell voran und soll auf lange Sicht Eclipse als UI5-IDE ablösen. 

Der Aufwand zur Entwicklung von Hybrid-Apps ist nicht verringert worden. Alle bekannten Build-Tools wie das Android SDK, Cordova, das Kapsel SDK u.\,a. müssen wie bisher lokal installiert und konfiguriert werden. Die Integration der Web IDE in externe Test- oder CI-Toolchains ist nicht möglich.

\subsubsection{Was benötigen wir?}
Wichtig sind möglichst wenig Kontextwechsel während der Entwicklung und damit eine hochintegrierte Entwicklungsumgebung. Das Werkzeug soll in den Hintergrund rücken und \ac{TDD} aktiv unterstützen. Wir wollen die gleiche Toolchain wie auf dem Integrationsserver nutzen.
Konkret benötigen wir:
\begin{itemize}
	\item Syntax-Hervorhebung der genutzten Sprache (JavaScript)
	\item Code-Vervollständigung der Frameworks (UI5)
	\item Integration von statischen Code-Analysen per ESlint
	\item QUnit-Testausführung per Karma
	\item Visualisierung der Code-Coverage
	\item Paketverwaltung per npm und Bower
	\item Grunt-Taskverwaltung
\end{itemize}
Eclipse beinhaltet im Auslieferungszustand nur wenig davon und die vorhandenen Funktionen wie Code-Vervollständigung in JavaScript-Projekten funktionieren nur selten. Manche der TDD-Funktionen ließen sich theoretisch per Plugin nachrüsten. Grunt als zentraler Task Runner aber \zB nicht. Da die vorhandene SAP-Integration allein Eclipse als IDE nicht rechtfertigt, wählen wir eine alternative IDE mit JavaScript- und TDD-Fokus.

\subsubsection{Lösung: WebStorm}
Gerade unter Web-Entwicklern genießt JetBrains WebStorm\footnote{\url{https://www.jetbrains.com/webstorm/}} einen hervorragenden Ruf: Die IDE ist speziell auf JavaScript-Projekte angepasst und bringt, neben zahlreichen Plugins, eine sinnvolle Vorkonfiguration mit. Schon aus früheren Projekten ist uns auch die Reaktionsfreude der zugrunde liegenden IntelliJ-Plattform bekannt. Wie können die bisher Eclipse-exklusiven Funktionen genutzt werden?

Die Syntax-Püfung von XML-Views ist über die entsprechenden XML-Definitionen (*.xsd) verfügbar. Diese sind Bestandteil des UI5 Development Toolkits for Eclipse, aber auch separat auf GitHub erhältlich\footnote{\url{https://github.com/jbmurray/UI5-WebStorm-Files}}. Das gleiche Repository enthält Bibliotheken, die sich für die UI5-Code-Vervollständigung in WebStorm nutzen lassen.
%\footnote{\url{http://scn.sap.com/community/developer-center/front-end/blog/2014/09/22/configuring-jetbrains-webstorm-for-ui5-development}}

Das ABAP Team Repository kann nicht integriert werden. Einchecken als \ac{BSP} aus WebStorm ist nur über den externen Aufruf einer Transaktion möglich. Für Fiori-Apps ist das verschmerzbar, da der CI-Server das finale Deployment übernimmt. Für hybride Apps spielt es ohnehin keine Rolle -- Installationspakete werden auf anderen Wegen verteilt, \zB über die App Stores oder eine \ac{MDM}-Lösung wie SAP Afaria.

\begin{figure}[h]
	\centering
	\includegraphics[width=1.25\textwidth]{WebStorm} 
	\caption[WebStorm mit TDD-Integration]{WebStorm mit TDD-Integration.}
	\label{fig:WebStorm}
\end{figure}

Webstorm integriert alle geforderten \ac{TDD}-Funktionen. \autoref{fig:WebStorm} zeigt die Ausführung von Karma-Tests in der \ac{IDE}. Test-Status und Code-Abdeckung werden direkt im Quellcode visualisiert. Grün markierte Zeilen wurden getestet, rote nicht. Um die testgetriebene Entwicklung in einer neuen WebStorm-Installation fortzusetzen sind folgende Schritte nötig:
\begin{enumerate}
	\item Klonen des Projekt-Repositories
	\item Installieren der Entwicklungs-Abhängigkeiten per \emph{npm install}
	\item Installieren der Frontend-Abhängigkeiten per \emph{bower install}
\end{enumerate}
Die Test-Konfiguration muss nicht angepasst werden -- WebStorm nutzt die, bereits als Code vorliegende, CI-Toolchain mit ihren Grunt-Tasks automatisch. Die Toolchain wird nicht nur abgespielt, sondern kann auch mit IDE-Unterstützung bearbeitet werden. WebStorm hat eigene Paketverwaltungen für Node.js und Bower, die sich entsprechend nutzen lassen.

Auch viele der weiteren Konfigurationsdateien werden automatisch erkannt. Die statische Code-Analyse erfolgt parallel mithilfe der eigenen ESLint-Regeln. Automatische Code-Formatierung orientiert sich an der IDE-übergreifenden \emph{.editorconfig}. Selbst die \emph{.gitignore}-Konfiguration findet bei der Nachverfolgung lokaler Änderungen Verwendung.

Insgesamt wird die Entwicklung durch WebStorm aktiv unterstützt. Alles was automatisiert werden kann, wird es auch. Die Benutzererfahrung ist deutlich besser als unter Eclipse, wo vieles nur langsam oder gar nicht funktioniert. Gegenüber der SAP Web IDE ist die Integration und Weiterentwicklungsmöglichkeit der CI-Toolchain sehr gelungen.


\subsection{UI5-App}
Abbildung \ref{fig:app_main_screen} zeigt den Hauptbildschirm der entwickelten UI5-Scrum-App. Sie ist nach dem typischen Master-Detail-Konzept gestaltet: In der linken Master-View wird erst ein Kunde ausgewählt, dann ein Projekt und Sprint. Anschließend zeigt die große Detail-View rechts die entsprechenden Sprint-Daten. Die Ansicht ist responsive, wird bei kleineren Bildschirmen also automatisch angepasst.

\begin{figure}
	\centering
	\includegraphics[width=0.95\textwidth]{app_main_screen} 
	\caption[UI5-App -- Hauptbildschirm]{UI5-App -- Hauptbildschirm mit Mock-Daten.}
	\label{fig:app_main_screen}
\end{figure}

Intern entspricht die App dem Model-View-Controller-Prinzip. Daraus ergibt sich folgende (gekürzte) Dateistruktur, die den aktuellen Best Practices entspricht:
\begin{itemize}
	\item app/
	\begin{itemize}
		\item i18n/
		\begin{itemize}
			\item messageBundle.properties
			\item messageBundle\_de\_DE.properties
		\end{itemize}		
%		\item img/
%		\begin{itemize}
%			\item background\_logo.png
%			\item favicon.ico
%		\end{itemize}
		\item view/
		\begin{itemize}
			\item Master.view.xml
			\item Master.controller.js
		\end{itemize}
		\item Component.js
		\item index.html
		\item MyRouter.js
	\end{itemize}
\end{itemize}

\subsubsection{Kapselung}
Als Startpunkt dient die \textit{index.html}, sie ermöglicht die Ausführung der App im Browser. Metatags geben dem Browser Hinweise zur korrekten Darstellung. Danach findet nur noch das Bootstrapping, also Laden, der UI5-Bibliotheken statt. Anschließend wird die App in einem Shell-Container gestartet.

Die eigentliche App ist unabhängig von einer HTML-Datei in der \textit{Component.js} definiert. Durch diese Kapselung ist die App portabel. Eine weitere Ausführungsumgebung ist \zB das Fiori Launchpad. In der Komponente sind allgemeine Informationen wie App-Name und Version gespeichert. Ebenso festgelegt ist die erste aufzurufende View.

\subsubsection{Routing}
Daneben enthält \textit{Component.js} die Routen-Konfiguration. Das Routing-Konzept ist Teil der Kapselung, da die Navigation innerhalb der App über einen eigenen Router (\textit{MyRouter.js}) stattfindet. In Ausführungsumgebungen mit mehreren UI5-Apps werden Konflikte vermieden.

Weiterer Vorteil: Über bestimmte URLs ist immer der gleiche App-Zustand erreichbar. Wie von anderen Webanwendungen gewohnt, kann über den Parameter \textit{/Kunden(Kunde='ABAT')/Projekte} immer die Projektübersicht des Kunden, in diesem Fall abat, aufgerufen werden. Die entsprechende Seite lässt sich bequem als Lesezeichen speichern.

\subsubsection{Internationalisierung}
Statische Texte sind in Resource Bundles wie \textit{messageBundle\_de.properties} definiert. Sie enthalten Schlüssel/Wert-Paare mit der entsprechenden Übersetzung in einer bestimmten Sprache und Landeskennung. Englische Standardtexte sind grundsätzlich in \textit{messageBundle.properties} gespeichert. 

In einer XML-View wird statischer Text wie \textit{i18n>masterTitle} automatisch mit dem Pendant aus dem i18n-Model ersetzt. Die aktuelle Sprache wird \zB aus dem Browser, dem Betriebssystem oder einem URL-Parameter bestimmt. UI5 sucht automatisch das passende Resource Bundle.

\subsubsection{OData}
Die Adresse des OData-Services ist ebenfalls in der \textit{Component.js} festgelegt. Während der Initialisierung werden die Metadaten geladen und als globales App-Model festgelegt. 
Anschließend werden die Elemente einer Liste \zB über \textit{items='{/Kunden}'} entsprechend der Einträge im Pfad des OData-Services angezeigt. 
Bei einem Klick auf den Kunden, wird dieser als aktueller \textit{BindingContext} gespeichert. Die Elemente der nachfolgenden Liste werden dann unter \textit{{gespeicherterKunde}/Projekte} gesucht. An dieser Stelle greifen UI5 und die Navigationseigenschaften des OData-Services ineinander.

\subsection{Akzeptanztests mit OPA5}
OPA5 basiert auf QUnit und ermöglicht Akzeptanztests für UI5-Apps. Es erleichtert die Test-Programmierung durch spezielle Selektoren für UI5-Controls und die Verwaltung von Asynchronität.

Analog zu den Testfall-Szenarien im \autoref{sec:tests} werden Tests per \textit{Given/When/Then} definiert. Die Formulierung aus den Testfällen lässt sich fast direkt in Code umsetzen, siehe Listing \ref{lst:opa5}. Voraussetzung hierfür ist die vorherige Definition zahlreicher eigener \textit{Actions} und \textit{Assertions}. 

Ein Beispiel ist die Assertion \textit{iShouldSeeTheCustomerList()}. Diese wird aus \textit{NavigationAssertions.js} nachgeladen und dem Framework über \textit{extendConfig()} bekannt gemacht. Innerhalb der nachgeladenen Datei ruft sie generellere Funktionen wie \textit{iClickOnAListItem(listName)} auf. Hierüber ist eine grundsätzliche Modularisierung der Tests möglich.

Nur für bestimmte Views nützliche Testfunktionen werden diesen über PageObjects zugeteilt und automatisch mitgeladen. Dies erleichtert die Strukturierung. Trotzdem wird das Testprojekt bei größerem Umfang schnell unübersichtlich. Die meisten Funktionen müssen selbst erstellt werden. Ein grafischer Editor, wie \zB Jubula ihn mitbringt, wäre von großem Vorteil.

Die OPA5-Tests können über eine HTML-Datei gestartet werden. Diese dient nur als Container, analog zum Start der normalen App-Komponente. Dort werden sie wie klassische QUnit-Tests in einem iFrame ausgeführt. In unserer Toolchain genügt eine JavaScript-Datei wie Listing \ref{lst:opa5}, die von Karma ausgeführt wird.

Die Implementierung des Mockservers ist bereits in \autoref{sec:karma} beschrieben. Er wird über den URL-Parameter \textit{?responderOn=true} aktiviert. Der Test läuft jetzt auch ohne Verbindung zum Backend.
%Die gekürzte Verzeichnisstruktur der Tests:
%\begin{itemize}
%	\item test/
%	 \begin{itemize}
%		\item actions/
%		\item arrangements/
%		\item assertions/
%		\item Navigations.js
%%		\item opaTests.qunit.html
%	\end{itemize}
%\end{itemize}

\begin{listing}[H]
	\inputminted[tabsize=2]{js}{src/opa5.js}
	\caption{OPA5-Test-Szenario 1}
	\label{lst:opa5}
\end{listing}

\chapter{Schlussfolgerungen}
\label{cha:Schlussfolgerungen}

%%%----------------------------------------------------------
%%%Anhang
%\appendix
%\include{anhang_a}	% Quelltext dieses Dokuments

%%%----------------------------------------------------------
\MakeBibliography
%%%----------------------------------------------------------

%%%Messbox zur Druckkontrolle
\include{messbox}

\end{document}
